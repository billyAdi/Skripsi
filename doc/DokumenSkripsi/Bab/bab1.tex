%versi 2 (8-10-2016) 
\chapter{Pendahuluan}
\label{chap:intro}
   
\section{Latar Belakang}
\label{sec:label}
Git merupakan perangkat lunak \textit{Version Control Systems}\cite{chacon2014pro}.\textit{Version control} adalah sistem yang merekam perubahan pada \textit{file} atau sekumpulan \textit{file} dari waktu ke waktu. Perubahan yang terjadi pada \textit{repository} dicatat oleh Git dalam bentuk histori \textit{commit}. Setiap \textit{commit} mengandung informasi mengenai perubahan yang terjadi pada \textit{repository}, waktu perubahan, dan orang yang melakukan perubahan. \textit{Database} pada \textit{git} tidak bersifat terpusat, melainkan terdistribusi. Setiap orang yang terlibat mempunyai \textit{database} lokal pada masing-masing komputer, sehingga pengelolaan perangkat lunak dapat dilakukan secara \textit{online} dan \textit{offline}.

JGit adalah \textit{library} Java murni yang mengimplementasikan Git \textit{version control systems}\cite{JGit}. JGit dikembangkan oleh Eclipse Foundation. JGit bersifat \textit{open source}. Dengan menggunakan JGit, operasi-operasi dalam Git dapat diakses melalui program Java. 

Selenium adalah seperangkat alat yang secara khusus digunakan untuk mengotomatisasi \textit{web browsers}\cite{Selenium}. Dengan menggunakan Selenium WebDriver, pengguna dapat memasukkan \textit{script} bahasa pemrograman tertentu untuk melakukan pengujian. Bahasa pemrograman yang didukung yaitu C\#, Java, Perl, PHP, Python, Ruby, dan JavaScript. Selenium WebDriver dapat melakukan pengujian pada Google Chrome \textit{browser},  Firefox \textit{browser}, Opera \textit{browser}, Safari \textit{browser}, Internet Explorer \textit{browser}, dan Microsoft Edge \textit{browser}.  

Apache Commons CLI menyediakan API untuk melakukan \textit{parsing} argumen Command Line yang dikirimkan ke program\cite{Apache_Commons_CLI}. Selain itu \textit{library} ini juga dapat mendefinisikan Command Line Option yang terdapat pada program. Command Line Option dapat digunakan untuk mengkonfigurasi program.

Dengan adanya histori \textit{commit} pada Git, perkembangan suatu proyek perangkat lunak dapat dipantau. 
\textit{Programmer} bisa belajar dari proses perkembangan perangkat lunak pada \textit{commit} sebelumnya dan bisa membuat perkembangan perangkat lunak menjadi lebih efisien. Akan tetapi, untuk perangkat lunak yang memiliki banyak \textit{commit}, pemantauan progres dapat memakan waktu yang lama. Karena itu proses perkembangan perangkat lunak perlu divisualisasikan. 
  
Pada skripsi ini, akan dibuat sebuah perangkat lunak yang dapat membangun animasi \textit{timelapse} dari pengembangan proyek perangkat lunak berbasis \textit{web}. Yang akan dibuat animasinya adalah halaman-halaman  \textit{web} dari suatu perangkat lunak berbasis \textit{web} yang terekam oleh Git. Perangkat lunak ini dibangun menggunakan bahasa Java. Perangkat lunak ini menggunakan antarmuka terminal/konsol. Masukan perangkat lunak diambil dari argumen Command Line. Perangkat lunak dibuat menggunakan bantuan \textit{library} JGit, Selenium WebDriver, dan Apache Commons CLI. \textit{Output} yang dihasilkan dari program ini adalah \textit{file} hasil animasi yang bertipe GIF.

\section{Rumusan Masalah}
\label{sec:rumusan}
Rumusan masalah dari skripsi ini adalah sebagai berikut:
\begin{enumerate}
	\item Bagaimana cara membangkitkan animasi \textit{timelapse} pada pengembangan proyek perangkat lunak berbasis web?
	\item Bagaimana cara mengimplementasikan aplikasi untuk membangkitkan \textit{timelapse} pada pengembangan proyek perangkat lunak berbasis web?
	\item Kesimpulan apa yang dapat diambil dari animasi \textit{timelapse} pada beberapa situs \textit{web} Open Source?	
\end{enumerate}

\section{Tujuan}
\label{sec:tujuan}
Tujuan dari skripsi ini adalah sebagai berikut:
\begin{enumerate}
	\item Mengetahui cara untuk membangkitkan animasi \textit{timelapse} pada pengembangan proyek perangkat lunak berbasis web.
	\item Mengetahui cara untuk mengimplementasikan aplikasi untuk membangkitkan \textit{timelapse} pada pengembangan proyek perangkat lunak berbasis web.
	\item Mendapatkan kesimpulan dari animasi \textit{timelapse} pada beberapa situs \textit{web} Open Source.
\end{enumerate}
	

\section{Batasan Masalah}
\label{sec:batasan}
Batasan masalah pada skripsi ini adalah sebagai berikut:
\begin{enumerate}
		\item Perangkat lunak ini hanya membangkitkan animasi \textit{timelapse} untuk perangkat lunak berbasis \textit{web} yang repositorinya terekam oleh Git. 
		%\item Masukan perangkat lunak berupa alamat direktori proyek perangkat lunak yang terekam oleh Git.
	    \item Jumlah halaman \textit{web} pada konfigurasi maksimal berjumlah empat halaman.
	    \item \textit{Setup} perangkat lunak dilakukan secara otomatis melalui \textit{terminal command}.
\end{enumerate}
\section{Metodologi}
\label{sec:metlit}
Metodologi penelitian yang digunakan dalam skripsi ini adalah sebagai berikut:
\begin{enumerate}
		\item Melakukan studi literatur tentang Git, JGit, Selenium WebDriver, dan Apache Commons CLI.
		\item Melakukan analisis penggunaan Selenium WebDriver dan JGit untuk membangkitkan animasi timelapse.
		\item Merancang perangkat lunak.
		\item Membangun perangkat lunak.
		\item Melakukan eksperimen dan pengujian pada perangkat lunak.
	\end{enumerate}

\section{Sistematika Pembahasan}
\label{sec:sispem}
Setiap bab dalam penelitian ini memiliki sistematika penulisan yang dijelaskan ke dalam poin-poin sebagai berikut:
\begin{enumerate}
		\item Bab 1: Pendahuluan, yaitu membahas mengenai gambaran umum penelitian ini. Berisi tentang latar belakang, rumusan masalah, tujuan, batasan masalah, metode penelitian, dan sistematika penulisan.
		\item Bab 2: Dasar Teori, yaitu membahas mengenai teori-teori yang mendukung berjalannya penelitian ini. Berisi tentang teori Git, JGit, Selenium WebDriver, dan Apache Commons CLI.
		\item Bab 3: Analisis, yaitu membahas mengenai analisa masalah. Berisi tentang analisis aplikasi sejenis, analisis penggunaan JGit dan Selenium WebDriver untuk membangkitkan animasi timelapse, analisis fitur aplikasi yang dibangun, dan prapengujian.
		\item Bab 4: Perancangan, yaitu membahas mengenai perancangan yang dilakukan sebelum melakukan tahapan implementasi. Berisi tentang perancangan kelas dan perancangan antarmuka dari perangkat lunak.
		\item Bab 5: Implementasi dan Pengujian, yaitu membahas mengenai implementasi dan pengujian aplikasi yang telah dilakukan. Berisi tentang implementasi dan hasil pengujian perangkat lunak.
		\item Bab 6: Kesimpulan dan Saran, yaitu membahas hasil kesimpulan dari keseluruhan penelitian ini dan saran-saran yang dapat diberikan untuk penelitian berikutnya.
	\end{enumerate}
