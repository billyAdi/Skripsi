\chapter{Kesimpulan dan Saran}
\label{chap:kesimpulan_dan_saran}

\section{Kesimpulan}
\label{sec:kesimpulan}
Dari hasil penelitian yang dilakukan, didapatkan kesimpulan-kesimpulan sebagai berikut:
\begin{itemize}
\item Animasi \textit{timelapse} pada pengembangan proyek perangkat lunak berbasis \textit{web} dapat dibangkitkan dengan bantuan Git. Dengan Git, bisa didapatkan halaman-halaman \textit{web} dari proyek perangkat lunak berbasis \textit{web}, kemudian \textit{screenshot} dari halaman-halaman tersebut digabung menjadi satu \textit{file} bertipe GIF.
\item  Aplikasi untuk membangkitkan \textit{timelapse} pada pengembangan proyek perangkat lunak berbasis \textit{web} dapat diimplementasi dengan bantuan \textit{library} JGit, Selenium WebDriver, dan Apache Commons CLI.

\item Program sudah berjalan dengan baik dan dapat membangkitkan animasi \textit{timelapse} dari pengembangan proyek perangkat lunak berbasis \textit{web}. Hal ini ditunjukkan dengan keberhasilan program dalam membangkitkan animasi pada beberapa situs \textit{web} saat dilakukan pengujian eksperimental.

%Hal ini ditunjukkan dengan kesesuaian \textit{output} program dengan \textit{output} yang diharapkan pada saat dilakukan pengujian fungsional.





%\item Berdasarkan hasil pengujian menggunakan berbagai macam WebDriver, dapat diambil kesimpulan bahwa program dapat membangkitkan animasi dengan baik saat menggunakan OperaDriver, FirefoxDriver, dan InternetExplorerDriver. Saat dilakukan pengujian menggunakan EdgeDriver, tampilan \textit{layout} dari halaman \textit{web} menjadi tidak rapih.

%\item Animasi \textit{timelapse} dari proyek pengembangan perangkat lunak berbasis \textit{web} dapat dibuat dengan bantuan \textit{library} JGit, Selenium WebDriver, dan Apache Commons CLI. 
%\item Program dapat membangkitkan animasi \textit{timelapse} dari proyek pengembangan perangkat lunak berbasis \textit{web}. 
%\item Untuk repositori yang sederhana seperti Netflix Open Source Center, tidak dibutuhkan \textit{option} \texttt{-before-capture} untuk membangkitkan animasi. Untuk repositori yang memerlukan setup basis data seperti Piktora, dibutuhkan \textit{option} \texttt{-before-capture} untuk membangkitkan animasi. Pada repositori Bootstrap, \textit{option} \texttt{-before-capture} dibutuhkan karena letak \textit{file} "index.html" yang tidak konsisten pada beberapa \textit{commit}.
%\item Berdasarkan hasil pengujian eksperimental, dapat diambil kesimpulan bahwa program dapat membangkitkan animasi \textit{timelapse} pada berbagai \textit{website} dan beberapa WebDriver. Saat dilakukan pengujian menggunakan EdgeDriver, tampilan \textit{layout} dari halaman menjadi tidak rapih. Karena itu tidak disarankan untuk menggunakan EdgeDriver untuk membangkitkan animasi. Program dapat membangkitkan animasi \textit{timelapse} pada repositori Bootstrap yang memiliki 8547 \textit{commit}. 


%masukin bab 5
%\item Berdasarkan hasil pengujian menggunakan berbagai macam situs \textit{web}, dapat diambil kesimpulan bahwa semakin banyak jumlah \textit{commit}, semakin besar ukuran \textit{file} GIF yang dihasilkan. Repositori dari situs \textit{web} Yelp Open Source memiliki 99 \textit{commit}, \textit{file} GIF yang dihasilkan berukuran 4.63 MB. Repositori dari situs \textit{web} IBM Open Source memiliki 263  \textit{commit}, \textit{file} GIF yang dihasilkan berukuran 15.2 MB. Repositori dari situs \textit{web} Netflix Open Source Software Center memiliki 393  \textit{commit}, \textit{file} GIF yang dihasilkan berukuran 17.5 MB. Repositori dari situs \textit{web} React  memiliki 570  \textit{commit}, \textit{file} GIF yang dihasilkan berukuran 18.2 MB. Repositori dari situs \textit{web} Bootstrap  memiliki 8547  \textit{commit}, \textit{file} GIF yang dihasilkan berukuran 160 MB.
 

%masukin bab 5
%\item Untuk repositori situs \textit{web} yang sederhana seperti Netflix Open Source Center, Bootstrap, IBM Open Source, React, dan Yelp Open Source, tidak dibutuhkan \textit{option} \texttt{-before-capture} untuk membangkitkan animasi. Untuk repositori yang memerlukan setup basis data seperti Piktora, dibutuhkan \textit{option} \texttt{-before-capture} untuk membangkitkan animasi. 
 
\end{itemize}

\section{Saran}
\label{sec:saran}
Dari hasil penelitian yang dilakukan, berikut adalah beberapa saran untuk pengembangan:
\begin{enumerate}
\item Saat ini \textit{output} dari program berupa \textit{file} bertipe GIF. \textit{File} bertipe GIF ini memiliki dua kekurangan, yaitu keterbatasan warna dan ukuran \textit{file} yang besar. GIF memakai sistem palet warna dan hanya terbatas pada 256 warna saja. \textit{File} hasil animasi dari situs \textit{web} Bootstrap berukuran cukup besar, yaitu sebesar 160 MB. Dalam pengembangan berikutnya, sebaiknya menggunakan format \textit{file} yang lebih modern seperti MP4 atau format lainnya. 

 
\item Saat ini ukuran \textit{screenshot} dari halaman \textit{web} bergantung pada resolusi layar. Pada beberapa situs \textit{web}, tinggi dari halaman \textit{web} melebihi tinggi dari resolusi layar sehingga \textit{screenshot} hanya menampilkan sebagian konten dari halaman \textit{web}. Dalam pengembangan berikutnya, sebaiknya program bisa mendapatkan \textit{screenshot} halaman \textit{web} secara keseluruhan.  

%\item Saat ini ukuran resolusi dari \textit{file} hasil animasi bergantung pada resolusi layar yang digunakan. Dalam pengembangan berikutnya, sebaiknya ditambahkan Command Line Option untuk mengatur ukuran resolusi dari \textit{file} hasil animasi.

 
\end{enumerate}