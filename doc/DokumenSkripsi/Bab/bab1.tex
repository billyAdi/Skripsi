%versi 2 (8-10-2016) 
\chapter{Pendahuluan}
\label{chap:intro}
   
\section{Latar Belakang}
\label{sec:label}


\section{Rumusan Masalah}
\label{sec:rumusan}
Rumusan masalah dari skripsi ini adalah sebagai berikut:
\begin{enumerate}
	\item Bagaimana cara membangkitkan animasi timelapse pada pengembangan proyek perangkat lunak?
	\item Bagaimana cara menampilkan animasi timelapse?
\end{enumerate}

\section{Tujuan}
\label{sec:tujuan}
Tujuan dari skripsi ini adalah sebagai berikut:
\begin{enumerate}
	\item Mengetahui cara untuk membangkitkan animasi timelapse pada pengembangan proyek perangkat lunak.
	\item Mengetahui cara untuk menampilkan animasi timelapse. 
\end{enumerate}

\section{Batasan Masalah}
\label{sec:batasan}

\section{Metodologi}
\label{sec:metlit}
Metodologi penelitian yang digunakan dalam skripsi ini adalah sebagai berikut:
\begin{enumerate}
		\item Melakukan studi literatur tentang Git, Selenium WebDriver, dan JGit.
		\item Melakukan analisis penggunaan Selenium WebDriver dan JGit untuk membangkitkan animasi timelapse.
		\item Merancang perangkat lunak.
		\item Membangun perangkat lunak.
		\item Melakukan eksperimen dan pengujian pada perangkat lunak.
	\end{enumerate}

\section{Sistematika Pembahasan}
\label{sec:sispem}
Setiap bab dalam penelitian ini memiliki sistematika penulisan yang dijelaskan ke dalam poin-poin sebagai berikut:
\begin{enumerate}
		\item Bab 1: Pendahuluan, yaitu membahas mengenai gambaran umum penelitian ini. Berisi tentang latar belakang, rumusan masalah, tujuan, batasan masalah, metode penelitian, dan sistematika penulisan.
		\item Bab 2: Dasar Teori, yaitu membahas mengenai teori-teori yang mendukung berjalannya penelitian ini. Berisi tentang ///////
		\item Bab 3: Analisis, yaitu membahas mengenai analisa masalah. Berisi tentang ////
		\item Bab 4: Perancangan, yaitu membahas mengenai perancangan yang dilakukan sebelum melakukan tahapan implementasi. Berisi tentang perancangan perangkat lunak pembangkit /textit{timelapse} proyek pengembangan perangkat lunak.
		\item Bab 5: Implementasi dan Pengujian, yaitu membahas mengenai implementasi dan pengujian aplikasi yang telah dilakukan. Berisi tentang implementasi dan hasil pengujian aplikasi.
		\item Bab 6: Kesimpulan dan Saran, yaitu membahas hasil kesimpulan dari keseluruhan penelitian ini dan saran-saran yang dapat diberikan untuk penelitian berikutnya.
	\end{enumerate}
