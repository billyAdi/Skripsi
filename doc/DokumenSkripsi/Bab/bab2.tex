%versi 2 (8-10-2016)

\lstdefinelanguage{plaintext}{
  sensitive=false,
  comment=[l]{//},
  morecomment=[s]{/*}{*/},
  identifierstyle=\color{black},
  morestring=[b]',
  morestring=[b]"
}

\lstset
{ 
    language=plaintext,
    basicstyle=\footnotesize,
    numbers=left,
    stepnumber=1,
    showstringspaces=false,
    tabsize=1,
    breaklines=true,
    breakatwhitespace=false,
    frame=leftline
}



\chapter{Landasan Teori}
\label{chap:teori}

Pada bab ini dibahas dasar teori yang mendukung berjalannya skripsi ini. Dasar teori yang dibahas yaitu Git, JGit, Selenium WebDriver, dan Apache Commons CLI.

\section{Git}
\label{sec:git} 
Git merupakan perangkat lunak \textit{Version Control Systems}. Pada subbab ini, dijelaskan mengenai \textit{Version Control Systems}, cara kerja Git, Git \textit{checkout}, dan operasi-operasi dasar pada Git. Subbab ini mengacu pada \cite{chacon2014pro}.   

\subsection{Version Control Systems}
\label{subsec:vcs}
\textit{Version Control Systems} adalah sistem yang merekam perubahan pada \textit{file} atau sekumpulan \textit{file} dari waktu ke waktu.\textit{Version Control Systems} biasanya digunakan  untuk merekam file yang berisi \textit{source code program}, tetapi pada kenyataannya \textit{Version Control Systems} dapat merekam hampir semua jenis file dalam komputer. Terdapat tiga jenis \textit{Version Control Systems}, yaitu: \textit{Local Version Control Systems}, \textit{Centralized Version Control Systems}, dan \textit{Distributed Version Control Systems}.

\subsubsection{Local Version Control Systems}
Metode \textit{version-controlled} yang banyak digunakan orang adalah dengan cara menyalin sekumpulan \textit{file} ke direktori lain. Namun cara tersebut rentan terhadap \textit{error}.
Misalnya, terdapat direktori A dan B, pengguna ingin mengubah \textit{file} yang terdapat pada direktori B, tetapi pengguna lupa kalau dia sedang berada di direktori A, maka pengguna mengubah \textit{file} pada direktori yang salah. Untuk mengatasi masalah tersebut, \textit{programmer} mengembangkan \textit{Local Version Control Systems}. 

\begin{figure}[H]
	\centering
		\includegraphics[scale=0.25]{Gambar/localvcs.png}
	\caption{Local version control\cite{chacon2014pro}.}
	\label{fig:localvcs}
\end{figure}

Gambar \ref{fig:localvcs} merupakan struktur dari \textit{Local Version Control Systems}. \textit{Database local Version Control Systems} ini tersimpan pada \textit{local directory} di komputer. \textit{Database} ini menyimpan perubahan \textit{file} ke dalam beberapa versi atau \textit{state}. \textit{Local Version Control}, dapat melakukan \textit{checkout} \textit{file} ke versi atau \textit{state} tertentu.   
 
\subsubsection{Centralized Version Control Systems}
\begin{figure}[H]
	\centering
		\includegraphics[scale=0.25]{Gambar/centralizedvcs.png}
	\caption{Centralized version control\cite{chacon2014pro}.}
	\label{fig:cvcs}
\end{figure}

\textit{Local Version Control} hanya menyimpan \textit{file} pada satu komputer saja. Muncul masalah baru ketika \textit{user} ingin berkolaborasi dengan \textit{user} lain. Untuk mengatasi masalah ini dikembangkan \textit{Centralized version control}. Gambar \ref{fig:cvcs} merupakan struktur dari \textit{Centralized Version Control Systems}. Dalam \textit{Centralized Control Version Systems} terdapat sebuah \textit{server} yang menyimpan setiap versi \textit{file}, dan klien yang dapat melakukan \textit{checkout} \textit{file}.

Sistem \textit{Centralized Version Control Systems} memiliki beberapa kelebihan. Setiap \textit{user}  dapat mengetahui pekerjaan yang dilakukan oleh \textit{user} lain. Administrator dapat lebih mudah mengontrol \textit{database} \textit{Centralized Version Control Systems} dibandingkan dengan \textit{database} \textit{Local Version Control Systems} dari setiap klien.      

Sistem \textit{Centralized Version Control Systems} memiliki kelemahan. Jika \textit{server} pusat \textit{Centralized Version Control Systems} mati , maka perubahan pada \textit{file} tidak bisa disimpan. Klien juga tidak dapat melakukan kolaborasi dengan klien lain. Jika \textit{harddisk} pada server rusak, maka semua versi \textit{file} akan hilang.  

\subsubsection{Distributed Version Control Systems}
\begin{figure}[H]
	\centering
		\includegraphics[scale=0.5]{Gambar/dvcs.png}
	\caption{Distributed version control\cite{chacon2014pro}.}
	\label{fig:dvcs}
\end{figure}
Gambar \ref{fig:dvcs} merupakan struktur dari \textit{Distributed Version Control Systems}. Dalam sebuah DVCS (seperti Git, Mercurial, Bazaar atau Darcs), klien tidak hanya melakukan \textit{checkout} untuk \textit{snapshot} terakhir setiap \textit{file}, namun klien juga memiliki salinan dari repositori tersebut. Dengan kata lain setiap klien memiliki \textit{version database local} pada komputernya. Jika server pusat mati, klien masih bisa melakukan kolaborasi dan klien manapun dapat mengirimkan kembali salinan repositori ke \textit{server}.

\subsection{Cara Kerja Git}
\label{subsec:cara_kerja_git}
Salah satu perbedaan antara Git dengan VCS lainnya adalah dalam cara Git memperlakukan datanya. Kebanyakan sistem \textit{Version Control Systems} lain menyimpan informasi sebagai daftar perubahan \textit{file}. Pada Gambar \ref{fig:deltas}, terdapat tiga \textit{file}.\textit{Version Control Systems} menyimpan \textit{file} A, B, dan C pada versi pertama saja. Untuk versi kedua dan seterusnya yang disimpan adalah perubahan pada setiap \textit{file}. Sistem ini disebut juga sebagai \textit{delta-based Version Control Systems}. 
\begin{figure}[H]
	\centering
		\includegraphics[scale=0.5]{Gambar/deltas.png}
	\caption{Menyimpan data sebagai \textit{snapshots} dari \textit{project}\cite{chacon2014pro}.}
	\label{fig:deltas}
\end{figure}


Berbeda dengan \textit{Version Control Systems} lainnya, Git memperlakukan datanya sebagai sebuah kumpulan \textit{snapshot} dari sebuah miniatur \textit{file system}. Setiap kali dilakukan \textit{commit}, git merekam \textit{state} dari sekumpulan \textit{file} dan menyimpanannya sebagai \textit{reference} \textit{snapshot} tersebut. Gambar \ref{fig:snapshots}, menunjukkan \textit{snapshots} dari \textit{file} A, B, dan C. Pada versi kedua, \textit{file} B tidak mengalami perubahan, sehingga \textit{file} yang disimpan adalah referensi \textit{file} B pada versi sebelumnya.
\begin{figure}[H]
	\centering
		\includegraphics[scale=0.5]{Gambar/snapshots.png}
	\caption{Menyimpan data sebagai perubahan terhadap versi dasar dari setiap \textit{file}\cite{chacon2014pro}.}
	\label{fig:snapshots}
\end{figure}

\subsubsection{State pada Git}
Terdapat tiga \textit{state} pada Git yaitu \textit{committed}, \textit{modified}, and \textit{staged}.\textit{Committed} adalah \textit{state} dimana data sudah disimpan di \textit{local database}. \textit{Modified} \textit{state} dimana terdapat perubahan pada \textit{file}, namun \textit{file} tersebut belum di \textit{commit} ke \textit{database}. \textit{Staged} adalah \textit{state} dimana \textit{file} telah ditandai untuk kemudian dilakukan commit.

Terdapat tiga bagian utama dari sebuah \textit{project} Git yaitu direktori Git, \textit{working directory}, dan \textit{staging area}. Direktori Git merupakan tempat dimana Git menyimpan \textit{metadata} dan \textit{object database} dari \textit{project}. \textit{Working tree} adalah suatu \textit{snapshot} dari \textit{project}. Sekumpulan \textit{file} ini diambil dari \textit{database} di direktori Git dan ditempatkan pada \textit{disk} untuk digunakan dan dimodifikasi. \textit{Staging} area adalah \textit{file} yang menyimpan informasi mengenai apa yang menjadi \textit{commit} selanjutnya. \textit{File staging area} terdapat pada direktori Git. Untuk lebih jelasnya, lihat Gambar \ref{fig:git_state}.

Alur kerja dari Git adalah sebagai berikut:
\begin{enumerate}
\item Melakukan modifikasi pada \textit{file}.
\item Menandai perubahan pada \textit{file} dan memindahkannya ke \textit{staging area}.
\item Mengambil \textit{file} dari \textit{staging area} dan menyimpan \textit{snapshot} ke direktori Git. Proses ini disebut dengan \textit{commit}.
\end{enumerate}  

\begin{figure}[H]
	\centering
		\includegraphics[scale=0.5]{Gambar/git_state.png}
	\caption{ \textit{Working tree}, \textit{Staging area}, dan Git direktori\cite{chacon2014pro}.}
	\label{fig:git_state}
\end{figure}

\subsubsection{Commit}
Commit merupakan sebuah \textit{snapshot} dari suatu \textit{file} atau direktori. \textit{Commit} menggambarkan \textit{state} dari \textit{working directory}. Gambar \ref{fig:snapshots} menunjukkan terdapat tiga \textit{file} pada versi/\textit{commit} keempat. Dimana terdapat \textit{file} A1, B1, dan C1 pada \textit{working directory}. \textit{File} A1, B1, dan C2  merupakan \textit{state} \textit{file} A, B, dan C pada \textit{commit} keempat. 

Git melakukan \textit{check-summed} pada \textit{commit} sebelum menyimpannya ke Git repositori. Mekanisme yang digunakan untuk melakukan \textit{check-summed} disebut dengan \textit{SHA-1 hash}. \textit{SHA-1 hash} terdiri dari empat puluh karakter heksadesimal(0-9 a-f). Nilai dari \textit{SHA-1 hash} dihitung berdasarkan isi dari \textit{working directory} atau struktur direktori Git.


\begin{lstlisting}[caption={Contoh histori commit dalam pengembangan perangkat lunak},label={lst:git_histori},language=plaintext]
C:\Users\user\Documents\GitHub\train-tracker-ellena-angelica>git log

commit b8aeacbd4743619b7b2d790d45bde26b899641e0 (HEAD -> master, origin/master, origin/HEAD)
Author: adamadamadamadamadam <adamnurmishwari@gmail.com>
Date:   Thu May 3 01:15:31 2018 +0700

    commitan terakhir. mastiin g buang memory sm batre

commit f836cc65bf6d50e274df54aa06c6fb529667aa06
Author: Evelyn Wijaya <evelynwijaya777@gmail.com>
Date:   Wed May 2 22:03:25 2018 +0700

    Update README.md

commit 2e1ce9a03a1f417326c3c6586503303cf6daf6b8
Author: Evelyn Wijaya <evelynwijaya777@gmail.com>
Date:   Wed May 2 22:01:10 2018 +0700

    Create README.md

commit 2f04488f9008745e8e6f67da33ffb2f6c2c9e747
Author: Evelyn Wijaya <evelynwijaya777@gmail.com>
Date:   Wed May 2 14:08:02 2018 +0700

    fix stasiun double

commit 7d8b66a9c6500de2753cdeac1084dc049c0c9f20
Author: Evelyn Wijaya <evelynwijaya777@gmail.com>
Date:   Wed May 2 13:32:21 2018 +0700

    fix stasiun double
    
\end{lstlisting}

Seperti yang diperlihatkan pada Listing \ref{lst:git_histori}, setiap \textit{commit} memiliki beberapa informasi. Baris pertama menunjukkan \textit{commit} ID yang berupa \textit{SHA-1 hash}. Pada baris ini, \textit{Master} menunjukkan \textit{branch} yang sedang aktif, \textit{master} juga merupakan \textit{pointer} ke \textit{commit} terakhir. \textit{Head} merupakan \textit{reference} ke \textit{branch master}. \textit{Origin}/\textit{master} dan \textit{origin}/\textit{HEAD} merupakan \textit{master} dan \textit{HEAD} pada \textit{remote repository}. Baris kedua menunjukkan orang yang melakukan \textit{commit} dan alamat emailnya. Baris ketiga menunjukkan waktu terjadinya \textit{commit}. Baris terakhir berisi deskripsi dari \textit{commit} tersebut.

\subsection{Operasi Dasar pada Git}
\label{subsec:operasi_dasar_git}
Pada subbab ini dijelaskan mengenai operasi dasar dalam Git dan sintaks-sintaksnya. Sintaks-sintaksnya ini dimasukkan pada Git \textit{command line}. Berikut ini adalah operasi-operasi dasar dalam Git:
\begin{enumerate}
\item Init\\
Operasi ini digunakan untuk membuat repositori lokal baru dengan nama tertentu. Bisa juga digunakan untuk merekam direktori yang sudah ada. Berikut adalah sintaks untuk melakukan operasi  \textit{init}:\\
\texttt{\$ git init [project-name]}  
\item Add\\
Operasi ini digunakan untuk menandai perubahan pada \textit{file} dan memindahkan \textit{file} tersebut ke \textit{staging area}. Operasi ini juga digunakan untuk menambahkan \textit{file} yang dipantau perubahannya. Berikut adalah sintaks untuk melakukan operasi add:\\
\texttt{\$ git add [file]}  
\item Commit\\
Operasi ini digunakan untuk merekam \textit{snapshot} atau \textit{state} \textit{file} atau sekumpulan \textit{file}. Operasi ini juga digunakan untuk memindahkan \textit{file} yang berada di \textit{stagging area} ke repositori Git. Berikut adalah sintaks untuk melakukan operasi \textit{commit}:\\
\texttt{\$ git commit -m "[descriptive message]"}  
\item Branch\\
Operasi ini digunakan untuk menampilkan semua \textit{branch} yang ada pada repositori Git, membuat \textit{branch} baru, dan menghapus \textit{branch}. Berikut adalah sintaks-sintaks untuk melakukan operasi \textit{branch}:\\
\texttt{\$ git branch}\\ 
\texttt{\$ git branch [branch-name]}\\
\texttt{\$ git branch -d [branch-name]}\\
\texttt{\$ git branch -D [branch-name]} 
\item Diff\\
Operasi ini digunakan untuk menampilkan perbedaan pada \textit{file} yang belum masuk \textit{staging area}, menampilkan perbedaan pada \textit{file} yang berada di \textit{staging area} dengan \textit{file} di \textit{commit} sebelumnya, dan perbedaan \textit{file} antara dua \textit{branch}.  Berikut adalah sintaks-sintaks untuk melakukan operasi \textit{diff}:\\
\texttt{\$ git diff} \\
\texttt{\$ git diff --staged}\\
\texttt{\$ git diff [first-branch]...[second-branch]}
\item Clone\\
Operasi ini digunakan untuk menyalin repositori Git yang berada di komputer lain atau suatu \textit{server}. Berikut adalah sintaks untuk melakukan operasi \textit{clone}:\\
\texttt{\$ git clone [url]}
\item Fetch\\
Operasi ini digunakan untuk mengambil data dari \textit{remote} repositori ke repositori lokal. Berikut adalah sintaks untuk melakukan operasi \textit{fetch}:\\
\texttt{\$ git fetch [bookmark]}
\item Merge\\
Operasi ini digunakan untuk menggabungkan \textit{branch} tertentu dengan \textit{branch} yang sedang aktif. Operasi ini juga digunakan untuk menggabungkan data yang diambil dari \textit{remote} repositori dengan data pada \textit{working directory}. Berikut adalah sintaks untuk melakukan operasi \textit{merge}:\\
\texttt{\$ git merge [branch]/[bookmark]}
\item Pull\\
Operasi ini adalah gabungan dari operasi \textit{fetch} dan \textit{merge}. Berikut adalah sintaks untuk melakukan operasi \textit{pull}:\\
\texttt{\$ git pull}
\item Push\\
Operasi ini digunakan untuk mengirim data pada reposipori Git lokal ke \textit{remote repository}.
Berikut adalah sintaks untuk melakukan operasi \textit{push}:\\
\texttt{\$ git push [alias] [branch]}
\item Checkout\\
Operasi ini digunakan untuk berpindah ke \textit{branch} atau \textit{commit} tertentu, setelah itu memperbarui \textit{file} pada \textit{working directory} berdasarkan \textit{branch} atau \textit{commit} tersebut. Berikut ini adalah sintaks-sintaks untuk operasi \textit{checkout}:\\
\texttt{\$ git checkout [SHA-1 commit]}\\
\texttt{\$ git checkout [branch-name]}
\item Log\\
Operasi ini digunakan untuk menampilkan semua histori \textit{commit} pada \textit{branch} yang sedang aktif. Berikut ini adalah sintaks untuk melakukan operasi \textit{log}:\\
\texttt{\$ git log}
\end{enumerate}
\subsection{Git Checkout}
\label{subsec:git_checkout}
Seperti yang sudah dijelaskan pada subbab \ref{subsec:operasi_dasar_git}, \textit{checkout} dapat digunakan untuk berpindah ke \textit{branch} atau \textit{commit} tertentu. Operasi \textit{checkout} dapat dilakukan menggunakan sintaks \texttt{\$ git checkout} diikuti dengan nama \textit{branch} atau \textit{SHA-1 hash}. Gambar \ref{fig:git_checkout} menunjukkan contoh \textit{checkout} pada \textit{commit}. Posisi awal \textit{HEAD} menunjuk pada \textit{branch master}, setelah dilakukan \textit{checkout} ke \textit{commit kedua}, posisi \textit{HEAD} menunjuk pada \textit{commit kedua}. \textit{Working directory} diperbarui berdasarkan \textit{state} pada \textit{commit} kedua. 

\textit{HEAD} yang menunjuk langsung ke suatu \textit{commit} disebut dengan \textit{detached HEAD}. Perubahan yang terjadi pada \textit{detached HEAD} tidak akan terekam oleh Git. Jika terdapat perubahan, kemudian dilakukan \textit{checkout commit} atau \textit{branch}, perubahan tersebut akan hilang. Tetapi, perubahan tersebut bisa disimpan dengan cara membuat \textit{branch} baru. Posisi \textit{HEAD} akan menunjuk pada \textit{branch} baru dan \textit{HEAD} sudah tidak lagi dalam keadaan \textit{detached HEAD}. 
\begin{figure}[H]
	\centering
		\includegraphics[scale=0.6]{Gambar/gitcheckoutcommit.png}
	\caption{\textit{Checkout} pada \textit{commit}}
	\label{fig:git_checkout}
\end{figure}

\section{JGit}
\label{sec:jgit}
JGit adalah \textit{library} Java murni yang mengimplementasikan Git \textit{version control systems}\cite{JGit}. Dengan menggunakan JGit, operasi-operasi dalam Git bisa dilakukan melalui program Java. Pada subbab berikut dijelaskan beberapa kelas dari \textit{library} JGit. Subbab ini mengacu pada \cite{JGit_java_doc}. 

\subsection{Repository}
\label{subsec:repository}
Kelas ini merepresentasikan repositori Git. Berikut ini adalah beberapa \textit{method} dalam kelas ini:
\begin{itemize}
\item public void create() throws IOException\\
Berfungsi untuk membuat repositori Git baru.
\item public void create(boolean bare) throws IOException\\
Berfungsi untuk membuat repositori Git baru. \\
Parameter: jika bernilai \textit{true} maka dibuat \textit{bare repository} (repositori tanpa \textit{working directory}). 
\item public String getBranch() throws IOException\\
Berfungsi untuk mendapatkan nama \textit{branch} yang ditunjuk oleh \textit{HEAD}, \textit{method} ini melempar \textit{IOException}.\\ 
Kembalian: nama dari \textit{branch} yang sedang aktif, contohnya \textit{master}.

\item public ObjectId resolve(String revstr) throws AmbiguousObjectException, IncorrectObjectTypeException, RevisionSyntaxException, IOException\\
Parameter: \textit{expression} dari \textit{git object references}. \textit{Method} ini melempar \textit{AmbiguousObjectException}, \textit{IncorrectObjectTypeException}, \textit{RevisionSyntaxException}, dan \textit{IOException}.\\
Kembalian: sebuah objek \textit{ObjectId}.
\end{itemize} 

\subsection{FileRepository}
\label{subsec:filerepository}
Kelas ini merupakan turunan dari kelas \textit{Repository}. Berikut ini adalah \textit{construtor} dari kelas ini:
\begin{itemize}
\item public FileRepository(String gitDir) throws IOException\\
\textit{Constructor} ini membuat repositori berdasarkan parameter, \textit{constructor} ini melempar IOException.\\
Parameter: lokasi dari \textit{repository metadata}, lokasi ini berupa \textit{path}.
\end{itemize}

\subsection{Git}
\label{subsec:Git}
Kelas ini menyediakan API yang mirip Git \textit{Command Line} untuk berinteraksi dengan repositori git. Berikut ini adalah \textit{constructor} dan beberapa \textit{method} dalam kelas ini:
\begin{itemize}
\item public Git(Repository repo)\\
\textit{Constructor} ini membuat objek Git yang digunakan untuk berinteraksi dengan repositori Git.
Parameter: objek \textit{Repository} yang digunakan untuk berinteraksi. Parameter tidak boleh bernilai \textit{null}. 

\item public static InitCommand init()\\
\textit{Method} ini mengembalikan objek \textit{command} untuk mengeksekusi operasi \textit{init}.\\
Kembalian: objek \textit{InitCommand} yang berfungsi untuk mengumpulkan parameter opsional dan akhirnya mengeksekusi operasi \textit{init}.

\item public AddCommand add()\\
\textit{Method} ini mengembalikan objek \textit{command} untuk mengeksekusi operasi \textit{add}.\\
Kembalian: objek \textit{AddCommand} yang berfungsi untuk mengumpulkan parameter opsional dan akhirnya mengeksekusi operasi \textit{add}.

\item public LogCommand log()\\
\textit{Method} ini mengembalikan objek \textit{command} untuk mengeksekusi operasi \textit{log}.\\
Kembalian: objek \textit{LogCommand} yang berfungsi untuk mengumpulkan parameter opsional dan akhirnya mengeksekusi operasi \textit{log}.

\item public CheckoutCommand checkout()\\
\textit{Method} ini mengembalikan objek \textit{command} untuk mengeksekusi operasi \textit{checkout}.\\
Kembalian: objek \textit{CheckoutCommand} yang berfungsi untuk mengumpulkan parameter opsional dan akhirnya mengeksekusi operasi \textit{checkout}.

\item public CommitCommand commit()\\
\textit{Method} ini mengembalikan objek \textit{command} untuk mengeksekusi operasi \textit{commit}.\\
Kembalian: objek \textit{CommitCommand} yang berfungsi untuk mengumpulkan parameter opsional dan akhirnya mengeksekusi operasi \textit{commit}.

\item public FetchCommand fetch()\\
\textit{Method} ini mengembalikan objek \textit{command} untuk mengeksekusi operasi \textit{fetch}.\\
Kembalian: objek \textit{FetchCommand} yang berfungsi untuk mengumpulkan parameter opsional dan akhirnya mengeksekusi operasi \textit{fetch}.

\item public PushCommand push()\\
\textit{Method} ini mengembalikan objek \textit{command} untuk mengeksekusi operasi \textit{push}.\\
Kembalian: objek \textit{PushCommand} yang berfungsi untuk mengumpulkan parameter opsional dan akhirnya mengeksekusi operasi \textit{push}.

\item public DiffCommand diff()\\
\textit{Method} ini mengembalikan objek \textit{command} untuk mengeksekusi operasi \textit{diff}.\\
Kembalian: objek \textit{DiffCommand} yang berfungsi untuk mengumpulkan parameter opsional dan akhirnya mengeksekusi operasi \textit{diff}.

\item public static CloneCommand cloneRepository()\\
\textit{Method} ini mengembalikan objek \textit{command} untuk mengeksekusi operasi \textit{clone}.\\
Kembalian: objek \textit{DiffCommand} yang berfungsi untuk mengumpulkan parameter opsional dan akhirnya mengeksekusi operasi \textit{clone}.

\item public MergeCommand merge()\\
\textit{Method} ini mengembalikan objek \textit{command} untuk mengeksekusi operasi \textit{merge}.\\
Kembalian: objek \textit{MergeCommand} yang berfungsi untuk mengumpulkan parameter opsional dan akhirnya mengeksekusi operasi \textit{merge}.

\item public PullCommand pull()\\
\textit{Method} ini mengembalikan objek \textit{command} untuk mengeksekusi operasi \textit{pull}.\\
Kembalian: objek \textit{PullCommand}.

\item public CreateBranchCommand branchCreate()\\
\textit{Method} ini mengembalikan objek \textit{command} untuk membuat \textit{branch} baru.\\
Kembalian: objek \textit{CreateBranchCommand}.

\item public public ListBranchCommand branchList()\\
\textit{Method} ini mengembalikan objek \textit{command} untuk menampilkan daftar \textit{branch}.\\
Kembalian: objek \textit{ListBranchCommand}.

\item public DeleteBranchCommand branchDelete()\\
\textit{Method} ini mengembalikan objek \textit{command} untuk menghapus \textit{branch}.\\
Kembalian: objek \textit{DeleteBranchCommand}.
\end{itemize}

\subsection{RevWalk}
\label{subsec:revwalk}
Kelas ini digunakan untuk menelusuri \textit{commit graph}. \textit{Instance} dari kelas ini hanya bisa melakukan  \textit{graph traversal} satu kali, untuk melakukan \textit{traversal} kedua dibutuhkan \textit{instance} baru atau memanggil \textit{method} reset(). Berikut ini adalah \textit{constructor} dan beberapa \textit{method} dalam kelas ini:
\begin{itemize}
\item public RevWalk(Repository repo)\\
\textit{Constructor} ini membuat objek \textit{revision walker} untuk suatu \textit{repository}.\\
Parameter: repositori yang digunakan untuk \textit{traversal}.  

\item public RevCommit parseCommit(AnyObjectId id)\\
Menempatkan \textit{reference} ke suatu \textit{commit} kemudian melakukan \textit{parsing} pada isi \textit{commit}.
Parameter: nama dari objek \textit{commit}.\\
Kembalian: \textit{reference} ke objek \textit{commit}.

\item public void sort(RevSort s)\\
Berfungsi untuk mengurutkan \textit{commit} berdasarkan metode dari parameter.\\
Parameter: metode untuk mengurutkan \textit{commit}.

\item public Iterator<RevCommit> iterator()\\
Berfungsi untuk mengembalikan \textit{iterator} yang bertipe \textit{RevCommit}.\\
Kembalian: \textit{iterator} dari \textit{RevCommit}.

\item public void markStart(RevCommit c) throws MissingObjectException, IncorrectObjectTypeException, IOException\\
Berfungsi untuk menandai \textit{commit} pertama untuk memulai \textit{traversal}. Method ini melempar MissingObjectException, IncorrectObjectTypeException, dan IOException.\\
Parameter: \textit{commit} awal yang digunakan untuk melakukan \textit{traversal}.

\item public final void reset()\\
Berfungsi untuk mengembalkan \textit{state} dari kelas ini ke \textit{state} semula, sehingga \textit{instance RevWalk} bisa digunakan lagi.
\end{itemize}

\subsection{RevCommit}
\label{subsec:revcommit}
Kelas ini merupakan \textit{reference} ke \textit{commit} yang ada di \textit{Directed Acyclic Graph}. Berikut ini adalah \textit{constructor} dan beberapa \textit{method} dari kelas ini:
\begin{itemize}
\item protected RevCommit(AnyObjectId id)\\
\textit{Constructor} ini membuat objek yang merupakan \textit{reference} ke suatu \textit{commit}.\\
Parameter: nama dari objek \textit{commit}.  

\item public final String getFullMessage()\\
Berfungsi untuk melakukan \textit{parsing} pada \textit{full commit message} dan mengubahnya ke \textit{string}.\\
Kembalian: \textit{string} hasil \textit{decode} dari \textit{commit message}.

\item public final String getShortMessage()\\
Berfungsi untuk melakukan \textit{parsing} pada \textit{commit message} dan mengubahnya ke \textit{string}, hanya baris pertama yang dikembalikan.\\
Kembalian: baris pertama \textit{string} hasil \textit{decode} dari \textit{commit message}.

\item public final String getName()\\
\textit{Method} ini mengembalikan \textit{SHA-1} dalam bentuk \textit{string}.
Kembalian: \textit{string SHA-1} dalam bentuk heksadesimal. 

\item public final PersonIdent getAuthorIdent()\\
Berfungsi untuk mendapatkan informasi mengenai \textit{author} yang melakukan \textit{commit}.\\
Kembalian: objek \textit{PersonIdent} yang memuat informasi tentang \textit{author}(nama dan \textit{email}) dan waktu dilakukannya \textit{commit}.
\end{itemize}

\subsection{PersonIdent}
\label{subsec:personident}
Kelas ini memberikan informasi mengenai \textit{author} dari suatu \textit{commit}. Berikut ini adalah beberapa \textit{method} dari kelas ini:
\begin{itemize}
\item public String getName()\\
Berfungsi untuk mengembalikan nama dari \textit{author} yang melakukan \textit{commit}.\\
Kembalian: nama dari \textit{author}.

\item public String getEmailAddress()\\
Berfungsi untuk mengembalikan alamat \textit{email} dari \textit{author} yang melakukan \textit{commit}.\\
Kembalian: alamat \textit{email} dari \textit{author}.

\item public Date getWhen()\\
Berfungsi mengembalikan waktu dilakukannya suatu \textit{commit} oleh \textit{author}.\\
Kembalian: sebuah \textit{timestamp}.
\end{itemize}


\section{Selenium WebDriver}
\label{sec:selenium_webdriver}
Selenium adalah kumpulan dari kakas perangkat lunak, dengan pendekatan yang berbeda pada setiap kakas dalam mendukung \textit{automation test}\cite{Selenium_doc}. Selenium mendukung bahasa pemrograman C\#, Java, Perl, PHP, Python, Ruby, dan JavaScript. Selenium terdiri dari beberapa kakas, yaitu Selenium 1(Selenium RC), Selenium 2(Selenium WebDriver), Selenium-Grid, dan Selenium IDE. Selenium RC merupakan proyek utama \textit{Selenium} untuk waktu yang lama, sebelum akhirnya bergabung dengan \textit{WebDriver} menjadi Selenium 2. Selenium RC bekerja dengan cara menginjeksi kode JavaScript ke \textit{browser} ketika \textit{browser} dimuat dan menggunakan JavaScript tersebut untuk menjalankan \textit{Application Under Test} dalam \textit{browser}. Selenium RC sekarang sudah \textit{deprecated} dan tidak digunakan lagi. Selenium Webdriver merupakan gabungan dari Selenium RC dan WebDriver. Selenium IDE merupakan kakas yang digunakan untuk mengembangkan Selenium \textit{test cases}.

WebDriver merupakan kakas untuk mengotomatisasi pengujian pada perangkat lunak web\cite{Selenium_doc}. WebDriver dapat berkomunikasi dengan \textit{browser} menggunakan \textit{native support} pada \textit{browser} untuk automasi. Setiap \textit{browser} memiliki WebDriver masing-masing. WebDriver yang terdapat pada SeleniumDriver antara lain ChromeDriver, FirefoxDriver/GeckoDriver, OperaDriver, InternetExplorerDriver, dan HtmlUnitDriver. 

Pada skripsi ini \textit{tools} Selenium yang digunakan hanya Selenium WebDriver. WebDriver yang digunakan adalah ChromeDriver. Bahasa yang digunakan adalah Java. Pada subbab berikut dijelaskan beberapa kelas dari \textit{library} Selenium WebDriver. Subbab ini mengacu pada \cite{Selenium_java_doc}.

\subsection{WebDriver}
\label{subsec:webdriver}
Kelas ini merupakan \textit{interface} utama yang digunakan untuk pengujian, kelas ini merepresentasikan \textit{web browser} yang ideal . Berikut ini adalah beberapa \textit{method} dalam kelas ini:
\begin{itemize}
\item void close()\\
Berfungsi untuk menutup \textit{window} pada \textit{browser}, jika \textit{window} yang sekarang merupakan satu-satunya \textit{window} yang terbuka maka \textit{browser} akan ditutup.
\item void quit()\\
Berfungsi untuk menutup driver dan semua \textit{window} yang sedang terbuka.
\item void get(String url)\\
Berfungsi untuk memuat halaman \textit{web} pada \textit{window} saat ini. \textit{Method} ini mengirim \textit{HTPP GET Request} untuk memuat halaman, dan \textit{method} ini akan melakukan \textit{blocking} sampai halaman \textit{web} selesai dimuat.\\
Parameter: alamat \textit{url} untuk memuat halaman \textit{web}.
\item String getTitle()\\
Berfungsi untuk mengembalikan judul dari halaman \textit{web} yang sedang aktif.\\
Kembalian: judul dari halaman \textit{web}.
\item String getCurrentUrl()\\
Berfungi untuk mendapatkan URL yang sedang aktif di \textit{browser}.\\
Kembalian: URL dari halaman \textit{web} yang sedang dimuat di \textit{browser}.
\end{itemize}

\subsection{WebElement}
\label{subsec:webelement}  
Kelas ini adalah \textit{Interface} yang  merupakan representasi dari elemen HTML. Berikut ini adalah beberapa \textit{method} yang dimiliki kelas ini:
\begin{itemize}
\item void click()\\
Berfungsi untuk mengklik suatu elemen HTML.
\item void submit()\\
Berfungsi untuk mengirimkan elemen \textit{form} ke \textit{remote server}. Fungsi ini akan melempar \textit{NoSuchElementException} jika elemen yang dikirim tidak berada di dalam \textit{form}. 
\item String getText()\\
Berfungsi untuk mendapatkan teks pada suatu elemen.\\
Kembalian: Teks yang \textit{visible} pada elemen.

\item void clear()\\
Berfungsi untuk menghapus teks pada elemen yang digunakan untuk memasukkan teks.
\item WebElement findElement(By by)\\
Berfungsi untuk mendapatkan \textit{WebElement} pertama menggunakan metode yang diberikan parameter. \textit{Method} ini akan melempar \textit{NoSuchElementException} jika \textit{WebElement} tidak ditemukan.\\
Kembalian: \textit{WebElement} pertama yang sesuai dengan mekanisme pencarian.\\
Parameter: mekanisme pencarian, bisa berupa pencarian dengan \textit{ID}, \textit{class}, dll.

\item List<WebElement> findElements(By by)\\
Berfungsi untuk mendapatkan semua \textit{WebElement} sesuai dengan mekanisme yang diberikan parameter.\\
Kembalian: \textit{list} dari \textit{WebElement}, atau \textit{list} kosong jika pencarian tidak ditemukan.\\
Parameter: mekanisme pencarian, bisa berupa pencarian dengan \textit{ID}, \textit{class}, dll.
\item void sendKeys(java.lang.CharSequence... keysToSend)\\
Berfungsi untuk mengirimkan kumpulan karakter/teks ke elemen \textit{input}. \textit{Method} ini akan melempar \textit{java.lang.IllegalArgumentException} jika parameter keysToSend bernilai \textit{null}.\\
Parameter: kumpulan karakter/teks yang dikirim ke elemen.

\item String getAttribute(String name)\\
Berfungsi untuk mendapatkan nilai dari \textit{attribute} suatu \textit{web element}.\\
Kembalian: nilai dari \textit{attribute} dari \textit{web element}.
\end{itemize} 

\subsection{OutputType}
\label{subsec:output_type}
Kelas ini merupakan \textit{interface} yang menentukan tipe \textit{output} pada \textit{screenshot}. Terdapat tiga konstanta untuk menentukan tipe \textit{output} pada \textit{screenshot}. Konstanta tersebut adalah sebagai berikut:
\begin{itemize}
\item static final OutputType<String> BASE64\\
Berfungsi untuk mendapatkan \textit{screenshot} dalam bentuk \textit{base64 data}.
\item static final OutputType<byte[]> BYTES\\
Berfungsi untuk mendapatkan \textit{screenshot} dalam bentuk \textit{raw bytes}.
\item static final OutputType<java.io.File> FILE\\
Berfungsi untuk mendapatkan \textit{screenshot} dalam bentuk \textit{temprorary file} yang akan dihapus setelah program keluar dari \textit{Java Virtual Machine}.
\end{itemize}


\subsection{TakesScreenshot}
\label{subsec:takes_screenshot}
Kelas ini merupakan \textit{interface} yang digunakan untuk mengambil \textit{screenshot}. Kelas ini hanya mempunyai satu method yaitu:
\begin{itemize}
\item <X> X getScreenshotAs(OutputType<X> target) throws WebDriverException\\
\textit{Method} ini berfungsi untuk mengambil \textit{screenshot} dan menyimpannya ke lokasi yang sudah ditentukan.\\
Kembalian: objek yang menyimpan informasi terkait \textit{screenshot} \\
Parameter: tipe \textit{output} yang diinginkan(lihat \ref{subsec:output_type}).

\end{itemize}


\section{Apache Commons CLI}
\label{subsec:apache_cli}
\textit{Library} Apache Commons CLI menyediakan API untuk menguraikan \textit{command-line options} yang dikirimkan ke program\cite{Apache_Commons_CLI}. Apache Commons CLI termasuk ke dalam salah satu \textit{project} Apache Commons. Tujuan utama dari \textit{project} Apache Commons adalah membuat dan melakukan \textit{maintain} pada komponen Java yang \textit{reusable}. Pada subbab berikut dijelaskan beberapa kelas dari \textit{library} Apache Commons CLI. Subbab ini mengacu pada \cite{Apache_java_doc}.

\subsection{CommandLineParser}
\label{subsec:commandlineparser}
Kelas ini merupakan \textit{interface}. Kelas yang mengimplementasikan \textit{interface} ini dapat menguraikan \textit{array of String} berdasarkan pada parameter/argumen yang diberikan. Berikut ini adalah beberapa \textit{method} yang dimiliki \textit{interface} ini: 
\begin{itemize}
\item CommandLine parse(Options options, String[] arguments) throws ParseException\\
Berfungsi untuk menguraikan argumen berdasarkan pada \textit{option} yang ditentukan. \textit{Method} ini melempar \textit{ParseException}.\\
Parameter: \textit{option} yang ditentukan, argumen \textit{command line}.\\
Kembalian: objek \textit{CommandLine}.

\item CommandLine parse(Options options, String[] arguments,boolean stopAtNonOption) throws ParseException\\
Berfungsi untuk menguraikan argumen berdasarkan pada \textit{option} yang ditentukan.\\
Parameter: \textit{option} yang ditentukan, argumen \textit{command line}, dan suatu \textit{boolean} yang menentukan apakah \textit{parsing} dihentikan jika terdapat argumen yang tidak valid. Jika bernilai \textit{true}, \textit{parsing} akan dihentikan dan semua argumen yang sudah diuraikan akan ditambahkan ke objek \textit{CommandLine}. Jika bernilai \textit{false}, akan dilempar \textit{ParseException} bila terdapat argumen yang tidak valid.
\\
Kembalian: objek \textit{CommandLine}.
\end{itemize}

\subsection{CommandLine}
\label{subsec:commandline}
Kelas ini merepresentasikan kumpulan argumen yang diuraikan terhadap \textit{options descriptor}.
Berikut ini adalah beberapa \textit{method} yang dimiliki kelas ini: 
\begin{itemize}
\item public String getOptionValue(String opt)\\
Berfungsi untuk mendapatkan nilai dari suatu \textit{option} berdasarkan parameter.\\
Parameter: nama dari \textit{option}.\\
Kembalian: nilai dari \textit{option}. Jika \textit{option} belum diatur, akan dikembalikan \textit{null}.
\item protected void addOption(Option opt)\\
Berfungsi untuk menambahkan \textit{option} ke \textit{command line}.\\
Parameter: objek \textit{option} yang ingin ditambahkan.
\item public boolean hasOption(String opt)\\
Berfungsi untuk menentukan apakah suatu \textit{option} sudah diatur.\\
Parameter: nama dari \textit{option}.\\
Kembalian: \textit{true} jika \textit{option} sudah diatur, \textit{false} jika \textit{option} belum diatur.
\item public Option[] getOptions()\\
Berfungsi untuk mengembalikan \textit{array} dari \textit{option} yang sudah diproses.\\ 
Kembalian: \textit{iterator} dari \textit{option} yang sudah diproses.
\item public Iterator<Option> iterator()\\
Berfungsi untuk mengembalikan \textit{iterator} dari \textit{option} yang sudah diproses.\\ 
Kembalian: \textit{array} dari \textit{option} yang sudah diproses.

\end{itemize}

\subsection{Options}
\label{subsec:options}
Kelas ini merepresentasikan kumpulan dari objek \textit{Option}, yang mendeskripsikan kemungkinan \textit{option} pada \textit{command line}. Berikut ini adalah beberapa \textit{method} yang dimiliki kelas ini: 
\begin{itemize}
\item public Options addOption(Option opt)\\
Berfungsi untuk menambahkan objek \textit{Option} ke kelas ini.
Parameter:\textit{option} yang akan ditambahkan.\\
Kembalian: hasil dari \textit{option} yang ditambahkan.

\item public Option getOption(String opt)\\
Berfungsi untuk mengembalikan objek \textit{Option} sesuai dengan nama yang diberikan paramater. 
Parameter: nama dari \textit{option} yang ingin dikembalikan.\\
Kembalian: objek \textit{option} berdasarkan parameter.

\end{itemize}


\subsection{Option}
\label{subsec:option}
Kelas ini mendeskripsikan sebuah \textit{command-line option}. Berikut ini adalah \textit{constructor} dan beberapa \textit{method} yang dimiliki kelas ini: 
\begin{itemize}
\item public Option(String opt, String description) throws IllegalArgumentException\\
\textit{Constructor} ini membuat objek \textit{option} sesuai dengan parameter yang diberikan.\textit{Constructor} ini melempar IllegalArgumentException.\\
Parameter: nama pendek \textit{option}, dan deskripsi dari \textit{option}.
\item public Option(String opt, boolean hasArg, String description) throws IllegalArgumentException\\
\textit{Constructor} ini membuat objek \textit{option} sesuai dengan parameter yang diberikan.\textit{Constructor} ini melempar IllegalArgumentException.\\
Parameter: nama pendek \textit{option}, suatu \textit{boolean} yang menentukan apakah \textit{option} membutuhkan argumen, dan deskripsi dari \textit{option}.
\item public Option(String opt, String longOpt, boolean hasArg, String description) throws IllegalArgumentException\\
{Constructor} ini membuat objek \textit{option} sesuai dengan parameter yang diberikan.\textit{Constructor} ini melempar IllegalArgumentException.\\
Parameter: nama pendek \textit{option}, nama panjang \textit{option}, suatu \textit{boolean} yang menentukan apakah \textit{option} membutuhkan argumen, dan deskripsi dari \textit{option}.
\item public boolean hasArg()\\
Berfungsi untuk mengetahui apakah suatu \textit{option} membutuhkan argumen.\\
Kembalian: \textit{true} jika \textit{option} ini membutuhkan argumen , \textit{false} jika \textit{option} ini tidak membutuhkan argumen.
\item public String getDescription()\\
Berfungsi untuk mendapatkan deskripsi dari suatu \textit{option}.\\
Kembalian: deskripsi dari \textit{option} ini.
\item public String getArgName()\\
Berfungsi untuk mendapatkan nama dari suatu \textit{option}.\\
Kembalian: nama dari argumen suatu \textit{option}
\item public String getLongOpt()\\
Berfungsi untuk mendapatkan nama panjang dari suatu \textit{option}.\\
Kembalian: nama panjang dari suatu \textit{option}.

\end{itemize}

\subsection{Option.Builder}
\label{subsec:optionbuilder}
Kelas ini merupakan \textit{nested class} dari kelas \textit{Option}. Kelas ini digunakan untuk membuat objek \textit{Option} dengan \textit{descriptive methods}. Berikut ini adalah beberapa \textit{method} yang dimiliki kelas ini: 
\begin{itemize}
\item public Option.Builder desc(String description)\\
Berfungsi untuk memberikan deskripsi pada \textit{option}.\\
Parameter: deskripsi dari \textit{option}.\\
Kembalian: objek \textit{Option.Builder} yang bisa digunakan untuk \textit{method chaining}.

\item public Option.Builder longOpt(String longOpt)\\
Berfungsi untuk memberikan nama panjang pada \textit{option}.\\
Parameter: nama panjang \textit{option}.\\
Kembalian: objek \textit{Option.Builder} yang bisa digunakan untuk \textit{method chaining}.

\item public public Option.Builder hasArg()\\
Berfungsi untuk menyatakan bahwa \textit{option} ini membutuhkan argumen.\\
Kembalian: objek \textit{Option.Builder} yang bisa digunakan untuk \textit{method chaining}.

\item public Option.Builder argName(String argName)\\
Berfungsi untuk memberi nama pada argumen.\\
Parameter: nama argumen.\\
Kembalian: objek \textit{Option.Builder} yang bisa digunakan untuk \textit{method chaining}.

\item public Option build()\\
Berfungi untuk membuat objek \textit{Option} berdasarkan nilai pada \textit{Option.Builder}.\\
Kembalian: objek \textit{Option}.

\end{itemize}


