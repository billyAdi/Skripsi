%versi 2 (8-10-2016)
\chapter{Landasan Teori}
\label{chap:teori}

Pada bab ini dibahas dasar teori yang mendukung berjalannya skripsi ini. Dasar teori yang dibahas yaitu Git, JGit, Selenium WebDriver, dan Apache Commons CLI.

\section{Git}
\label{sec:git} 
%Kalimat pertama: bab 1 blm di reference
Seperti yang telah dijelaskan pada Bab 1, Git merupakan perangkat lunak \textit{Version Control System}. Pada subbab ini, dijelaskan mengenai \textit{Version Control System}, sejarah singkat Git, dan operasi-operasi dasar pada Git.   

\subsection{Version Control System}%gambarnya diambil dari ebook
\textit{Version Control System} adalah sistem yang merekam perubahan pada \textit{file} atau sekumpulan \textit{file} dari waktu ke waktu\cite{chacon2014pro}.\textit{Version Control System} biasanya digunakan  untuk merekam file yang berisi \textit{source code program}, tetapi pada kenyataannya \textit{Version Control System} dapat merekam hampir semua jenis file dalam komputer. Terdapat tiga jenis \textit{Version Control System}, yaitu: \textit{local Version Control System}, \textit{centralized Version Control System}, dan \textit{distributed Version Control System}.

\subsubsection{Local Version Control System}%belum dicantumin sumber gambar
\subsubsection{Centralized Version Control System}%belum dicantumin sumber gambar
\subsubsection{Distributed Version Control System}%belum dicantumin sumber gambar



\subsection{Operasi Dasar pada Git}
\subsubsection{Init}
\subsubsection{Add}
\subsubsection{Commit}
\subsubsection{Clone}
\subsubsection{Fetch}
\subsubsection{Merge}
\subsubsection{Pull}
\subsubsection{Push}
\subsubsection{Checkout}
\subsubsection{Push}
\subsubsection{Branch}
\subsubsection{Diff}

\section{JGit}
\subsection{Porcelain API}
\subsection{Plumbing API}

\section{Selenium WebDriver}


\section{Apache Commons CLI}
