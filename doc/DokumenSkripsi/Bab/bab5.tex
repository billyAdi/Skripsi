
\lstdefinelanguage{plaintext}{
  sensitive=false,
  comment=[l]{//},
  morecomment=[s]{/*}{*/},
  identifierstyle=\color{black},
  morestring=[b]',
  morestring=[b]"
}

\lstset
{ 
    language=plaintext,
    basicstyle=\footnotesize,
    numbers=left,
    stepnumber=1,
    showstringspaces=false,
    tabsize=1,
    breaklines=true,
    breakatwhitespace=false,
    frame=leftline
}

\chapter{Implementasi dan Pengujian}
\label{chap:implementasidanpengujian}
Pada bab dijelaskan mengenai implementasi perangkat lunak dan pengujian perangkat lunak. Bagian implementasi berisi tentang lingkungan implementasi dan hasil implementasi. Bagian pengujian berisi tentang pengujian fungsional dan pengujian eksperimental. 

\section{Implementasi}
\label{sec:implementasi}
\subsection{Lingkungan Implementasi}
\label{subsec:lingkunganimplementasi5}
Implementasi dari perangkat lunak dilakukan pada sebuah laptop. Berikut adalah spesifikasi laptop dan perangkat lunak yang digunakan untuk prapengujian:
\begin{itemize}
\item Processor: Intel Core i3 4030U
\item RAM: 6GB
\item Sistem Operasi: Windows 10 pro 64-bit
\item Versi Apache HTTP Server: 2.4.29
\item Versi MySQL Server: 5.5.5
\item Versi Netbeans: 8.1
\item Versi Google Chrome: 73.0.3683.86
\end{itemize}

\subsection{Hasil Implementasi}
\label{subsec:lingkunganimplementasi}
Hasil dari implementasi adalah sebuah perangkat berbasis terminal yang dapat membangkitkan animasi \textit{timelapse} pada pengembangan proyek perangkat lunak berbasis \textit{web}. Kode program dari perangkat lunak dapat dilihat pada Lampiran \ref{lamp:A}. Setelah dijalankan, perangkat lunak akan menghasilkan dua \textit{output} yaitu, status pada terminal dan \textit{file} hasil animasi bertipe GIF.
\begin{enumerate}
\item \textbf{Status pada Terminal}\\
Setelah berhasil membangkitkan animasi \textit{timelapse}, perangkat lunak menampilkan status pada terminal seperti yang diperlihatkan pada Listing \ref{lst:status_berhasil}. Baris 5 menunjukkan bahwa animasi \textit{timelapse} berhasil dibangkitkan. Pesan pada baris 1-4 muncul saat ChromeDriver membuka dan mulai mengontrol Chrome \textit{browser}.

\begin{lstlisting}[caption={Status pesan pada terminal saat program berhasil membangkitkan animasi \textit{timelapse}.},label={lst:status_berhasil},language=plaintext]
Starting ChromeDriver 2.42.591088 (7b2b2dca23cca0862f674758c9a3933e685c27d5) on port 16446
Only local connections are allowed.
Feb 24, 2019 3:26:25 PM org.openqa.selenium.remote.ProtocolHandshake createSession
INFO: Detected dialect: OSS
Animasi timelapse berhasil dibuat
\end{lstlisting}

\item \textbf{\textit{File} GIF Hasil Animasi}\\
Selain menghasilkan status pada terminal, program juga akan menghasilkan sebuah \textit{file} GIF hasil animasi.
Gambar \ref{fig:c1} - Gambar \ref{fig:c12} menunjukkan \textit{screenshot} setiap \textit{commit} yang terdapat pada \textit{file} GIF hasil animasi dari proyek Piktora. Piktora memiliki 58 \textit{commit}, sehingga terdapat 58 \textit{screenshot}. 

\begin{figure}[H]
	
		\includegraphics[scale=0.3]{Gambar/Untitled-1.png}
	\caption{\textit{Screenshot} proyek Piktora pada \textit{commit} 315d374 (31 Oktober 2016) - \textit{commit} 5c59916 (8 November 2016).}
	\label{fig:c1}
\end{figure}


\begin{figure}[H]
	
		\includegraphics[scale=0.3]{Gambar/Untitled-2.png}
	\caption{\textit{Screenshot} proyek Piktora pada \textit{commit} 7738380 (8 November 2016) - \textit{commit} 3caf535 (15 November 2016).}
	\label{fig:c2}
\end{figure}

\begin{figure}[H]
	
		\includegraphics[scale=0.3]{Gambar/Untitled-3.png}
	\caption{\textit{Screenshot} proyek Piktora pada \textit{commit} c5eb3b6 (15 November 2016) - \textit{commit} 3eb7af8 (21 November 2016).}
	\label{fig:c3}
\end{figure}

\begin{figure}[H]
	
		\includegraphics[scale=0.3]{Gambar/Untitled-4.png}
	\caption{\textit{Screenshot} proyek Piktora pada \textit{commit} e87e84b (22 November 2016) - \textit{commit} f0f7270 (23 November 2016).}
	\label{fig:c4}
\end{figure}

\begin{figure}[H]
	
		\includegraphics[scale=0.3]{Gambar/Untitled-5.png}
	\caption{\textit{Screenshot} proyek Piktora pada \textit{commit} 57a239b (23 November 2016) - \textit{commit} 0fcd958 (28 November 2016).}
	\label{fig:c5}
\end{figure}

\begin{figure}[H]
	
		\includegraphics[scale=0.3]{Gambar/Untitled-6.png}
	\caption{\textit{Screenshot} proyek Piktora pada \textit{commit} add3974 (28 November 2016) - \textit{commit} 0fe9aaf (29 November 2016).}
	\label{fig:c6}
\end{figure}

\begin{figure}[H]
	
		\includegraphics[scale=0.3]{Gambar/Untitled-7.png}
	\caption{\textit{Screenshot} proyek Piktora pada \textit{commit} f2326dd (29 November 2016) - \textit{commit} c4e9576 (2 Desember 2016).}
	\label{fig:c7}
\end{figure}

\begin{figure}[H]
	
		\includegraphics[scale=0.3]{Gambar/Untitled-8.png}
	\caption{\textit{Screenshot} proyek Piktora pada \textit{commit} 02d04f1 (5 Desember 2016) - \textit{commit} eb49c2b (6 Desember 2016).}
	\label{fig:c8}
\end{figure}


\begin{figure}[H]
	
		\includegraphics[scale=0.3]{Gambar/Untitled-9.png}
	\caption{\textit{Screenshot} proyek Piktora pada \textit{commit} ace1988 (6 Desember 2016) - \textit{commit} c83f4aa (15 Desember 2016).}
	\label{fig:c9}
\end{figure}

\begin{figure}[H]
	
		\includegraphics[scale=0.3]{Gambar/Untitled-10.png}
	\caption{\textit{Screenshot} proyek Piktora pada \textit{commit} 57f5ea4 (15 Desember 2016) - \textit{commit} 1880a88 (5 Januari 2017).}
	\label{fig:c10}
\end{figure}

\begin{figure}[H]
	
		\includegraphics[scale=0.3]{Gambar/Untitled-11.png}
	\caption{\textit{Screenshot} proyek Piktora pada \textit{commit} 286aa78 (16 Januari 2017) - \textit{commit} 38711f0 (17 April 2017).}
	\label{fig:c11}
\end{figure}


\begin{figure}[H]
	
		\includegraphics[scale=0.3]{Gambar/Untitled-12.png}
	\caption{\textit{Screenshot} proyek Piktora pada \textit{commit} 9f041ef (15 Mei 2017) - \textit{commit} 89000be (12 Januari 2018).}
	\label{fig:c12}
\end{figure}


\end{enumerate}
\section{Pengujian}
\label{sec:pengujian}

\subsection{Pengujian Fungsional}
\label{sec:pengujian_fungsional} 
Pengujian ini dilakukan dengan tujuan untuk mengetahui apakah Command Line Option yang terdapat pada program sudah berjalan dengan baik. Option yang terdapat pada program dapat dilihat pada subbab \ref{sec:analisis_fitur_aplikasi}. Lingkungan pengujian fungsional sama dengan lingkungan implementasi yang terdapat pada subbab \ref{subsec:lingkunganimplementasi5}. Pengujian dilakukan pada proyek Piktora dengan menggunakan berbagai macam tes kasus. Histori \textit{commit} proyek Piktora dapat dilihat pada subbab \ref{sec:prapengujian}. Hasil pengujian fungsional dapat dilihat pada Tabel \ref{table:hasil_pengujian1} dan Tabel \ref{table:hasil_pengujian2}.


\begin{table}[htbp]
	\centering
	\caption{Tabel pengujian fungsional}
	
		\begin{tabular}{|p{0.3cm}| p{5 cm}| p{7.3 cm}| p{3 cm}|} \hline
		No & Tes Kasus	& \textit{Output} yang diharapkan & \textit{Output} program \\ \hline
		1. & \texttt{-project-path} C:/xampp/htdocs/Piktora/.git & Program berhasil membangkitkan animasi berdasarkan path yang diberikan dan mengeluarkan status pesan "Animasi timelapse berhasil dibuat" & sesuai \\ \hline
		2. & \texttt{-project-path} C:/xampp/htdocs  & Program gagal membuat animasi dan mengeluarkan pesan error "Path proyek tidak valid"  & sesuai \\ \hline
		3. & \texttt{-capture-url} http://localhost & Program mengambil screenshot berdasarkan alamat yang diberikan. Tampilan dari \textit{file} hasil animasi dapat dilihat pada Gambar \ref{fig:capture1}  & sesuai	\\ \hline
		4. & \texttt{-capture-url} http://localhost http://localhost/about & Program mengambil screenshot berdasarkan alamat-alamat yang diberikan. Tampilan dari \textit{file} hasil animasi dapat dilihat pada Gambar \ref{fig:capture2}  & sesuai \\ \hline
		5. & \texttt{-capture-url} http://localhost http://localhost/about http://localhost/projects & Program mengambil screenshot berdasarkan alamat-alamat yang diberikan. Tampilan dari \textit{file} hasil animasi dapat dilihat pada Gambar \ref{fig:capture3}   & sesuai\\ \hline
		6. & \texttt{-capture-url} http://localhost http://localhost/about http://localhost/projects http://localhost/contact & Program mengambil screenshot berdasarkan alamat-alamat yang diberikan. Tampilan dari \textit{file} hasil animasi dapat dilihat pada Gambar \ref{fig:capture4}  & sesuai \\ \hline
		7. & \texttt{-capture-url} http://localhost http://localhost/about http://localhost/projects http://localhost/contact http://localhost/projects/1& Program gagal membuat animasi dan mengeluarkan pesan error "Jumlah url yang akan dicapture maksimal 4" & sesuai \\ \hline
		8. & \texttt{-capture-url} http://localhost/project& Program gagal membuat animasi dan mengeluarkan pesan error "Capture url tidak valid"  & sesuai \\ \hline
		9. & \texttt{-seconds-per-commit} 2 & Program berhasil membangkitkan animasi dan menghasilkan \textit{file} GIF yang memiliki durasi 116 detik & sesuai \\ \hline
		10. & \texttt{-seconds-per-commit} 0 & Program gagal membuat animasi dan mengeluarkan pesan error "Seconds per commit harus lebih besar dari 0" & sesuai \\ \hline
		11. & \texttt{-seconds-per-commit} 656 & Program gagal membuat animasi dan mengeluarkan pesan error "Seconds per commit harus kurang dari sama dengan  655" & sesuai \\ \hline
		12. & \texttt{-before-capture} "php script\_piktora.php" & Program menjalankan \textit{terminal command} sebelum mengambil \textit{screenshot} & sesuai  \\ \hline
		13. & \texttt{-before-capture} "php " & Program gagal membuat animasi dan mengeluarkan pesan error "Terminal Command Tidak Valid" & sesuai \\ \hline
		14. & \texttt{-title} Piktora  & Program berhasil membangkitkan animasi dan menghasilkan \textit{file} GIF, dimana di dalam \textit{file} tersebut terdapat judul yang terletak di pojok kiri bawah(lihat Gambar \ref{fig:title}). & sesuai \\ \hline
		
		
		\end{tabular}
	\label{table:hasil_pengujian1}
\end{table}


\begin{table}[htbp]
	\centering
	\caption{Tabel pengujian fungsional}
	
		\begin{tabular}{|p{0.3cm}| p{5 cm}| p{7.3 cm}| p{3 cm}|} \hline
		No & Tes Kasus	& \textit{Output} yang diharapkan & \textit{Output} program \\ \hline
		15. & \texttt{-logo} Logo-UNPAR.png & Program berhasil membangkitkan animasi dan menghasilkan \textit{file} GIF, dimana di dalam \textit{file} tersebut logo yang terletak di pojok kanan bawah (lihat Gambar \ref{fig:logo}) & sesuai  \\ \hline
		16. & \texttt{-logo} script\_piktora.php & Program gagal membuat animasi dan mengeluarkan pesan error "Path gambar tidak valid" & sesuai  \\ \hline
		17. & \texttt{-branch} master & Program membangkitkan animasi pada \textit{branch} master & sesuai  \\ \hline
		18. & \texttt{-branch} piktora  & Program gagal membuat animasi dan mengeluarkan pesan error "Branch tidak valid" & sesuai  \\ \hline
		19. & \texttt{-start-commit} 9b0a302 & Program membangkitkan animasi dimulai dari commit 9b0a302 & sesuai  \\ \hline		
		
		20. & \texttt{-start-commit} 9b0a30 & Program gagal membuat animasi dan mengeluarkan pesan error "Panjang commit ID awal harus berada di antara 7-10 karakter" & sesuai  \\ \hline
		21. & \texttt{-start-commit} 9b0a3023ac8 & Program gagal membuat animasi dan mengeluarkan pesan error "Panjang commit ID awal harus berada di antara 7-10 karakter" & sesuai  \\ \hline
		22. & \texttt{-stop-commit} 6a085c1  & Program membangkitkan animasi dimulai dari \textit{commit} 315d374 sampai dengan \textit{commit} 6a085c1  & sesuai  \\ \hline
		23. & \texttt{-stop-commit} 6a085c & Program gagal membuat animasi dan mengeluarkan pesan error "Panjang commit ID akhir harus berada di antara 7-10 karakter" & sesuai  \\ \hline
		24. & \texttt{-stop-commit} 6a085c1c379 & Program gagal membuat animasi dan mengeluarkan pesan error "Panjang commit ID akhir harus berada di antara 7-10 karakter"  & sesuai  \\ \hline
		25. & \texttt{-start-commit} 9b0a302 -stop-commit 6a085c1 & Program membangkitkan animasi dimulai dari \textit{commit} 9b0a302 sampai dengan \textit{commit} 6a085c1 & sesuai  \\ \hline
		26. & \texttt{-start-commit} 6a085c1 -stop-commit 9b0a302 & Program gagal membuat animasi dan mengeluarkan pesan error "Commit ID awal dan akhir terbalik"  & sesuai\\ \hline
		27. & \texttt{-start-commit} 9b0a302  -stop-commit 9b0a302 & Program gagal membuat animasi dan mengeluarkan pesan error "Commit ID awal dan akhir tidak boleh sama" & sesuai  \\ \hline
\end{tabular}
	\label{table:hasil_pengujian2}
\end{table}
\ \\
\ \\
\begin{figure}[H]
	\centering
		\includegraphics[scale=0.3]{Gambar/title.png}
	\caption{Salah satu \textit{commit} yang terdapat pada \textit{file} hasil animasi. Terdapat judul dibagian pojok kiri bawah.}
	\label{fig:title}
\end{figure}


\begin{figure}[H]
	\centering
		\includegraphics[scale=0.3]{Gambar/logo.png}
	\caption{Salah satu \textit{commit} yang terdapat pada \textit{file} hasil animasi. Terdapat logo dibagian pojok kanan bawah.}
	\label{fig:logo}
\end{figure}

\begin{figure}[H]
	\centering
		\includegraphics[scale=0.3]{Gambar/capture1.png}
	\caption{Salah satu \textit{commit} pada \textit{file} hasil animasi jika terdapat satu argumen \texttt{-capture-url}.}
	\label{fig:capture1}
\end{figure}



\begin{figure}[H]
	\centering
		\includegraphics[scale=0.3]{Gambar/capture2.png}
	\caption{Salah satu \textit{commit} pada \textit{file} hasil animasi jika terdapat dua argumen \texttt{-capture-url}.}
	\label{fig:capture2}
\end{figure}



\begin{figure}[H]
	\centering
		\includegraphics[scale=0.3]{Gambar/capture3.png}
	\caption{Salah satu \textit{commit} pada \textit{file} hasil animasi jika terdapat tiga argumen \texttt{-capture-url}.}
	\label{fig:capture3}
\end{figure}


\begin{figure}[H]
	\centering
		\includegraphics[scale=0.3]{Gambar/capture4.png}
	\caption{Salah satu \textit{commit} pada \textit{file} hasil animasi jika terdapat empat argumen \texttt{-capture-url}.}
	\label{fig:capture4}
\end{figure}




\subsection{Pengujian Eksperimental}
\label{sec:pengujian_eksperimental} 
Pengujian eksperimental ini dibagi menjadi dua bagian. Pengujian eksperimental bagian pertama akan menguji program menggunakan proyek Piktora dengan WebDriver yang berbeda. WebDriver yang digunakan untuk pengujian ini yaitu FirefoxDriver, EdgeDriver, OperaDriver, dan InternetExplorerDriver. Pengujian eksperimental bagian kedua akan menguji progam dengan \texttt{website} Bootstrap dan Netflix Open Source Software Center. Berikut adalah rincian dari pengujian eksperimental:

\begin{enumerate}
\item \textbf{Pengujian Proyek Piktora dengan FirefoxDriver}\\
Pengujian pada proyek Piktora dilakukan menggunakan FirefoxDriver. Pada saat melakukan pengujian, dilakukan sedikit perubahan kode program pada kelas BrowserController baris ke-46 (lihat Lampiran \ref{lamp:A}). \textit{Object} bertipe WebDriver dinisialisasi menggunakan \textit{object} bertipe FirefoxDriver. Versi Firefox \textit{browser} yang digunakan untuk pengujian adalah 66.0.2. Berikut ini adalah \textit{Option} yang digunakan untuk menguji program:
\begin{itemize}
\item \texttt{-project-path} C:/xampp/htdocs/Piktora/.git
\item \texttt{-capture-url} http://localhost
\item \texttt{-before-capture} "php script\_piktora.php"
\end{itemize}
Program berhasil membangkitkan animasi \textit{timelapse} pada proyek Piktora menggunakan FirefoxDriver.


\item \textbf{Pengujian Proyek Piktora dengan OperaDriver}\\
Pengujian pada proyek Piktora dilakukan menggunakan OperaDriver. Pada saat melakukan pengujian, dilakukan sedikit perubahan kode program pada kelas BrowserController baris ke-46 (lihat Lampiran \ref{lamp:A}). \textit{Object} bertipe WebDriver dinisialisasi menggunakan \textit{object} bertipe OperaDriver. Versi Opera \textit{browser} yang digunakan untuk pengujian adalah 60.0.3255.27 (portable version). Berikut ini adalah \textit{Option} yang digunakan untuk menguji program:
\begin{itemize}
\item \texttt{-project-path} C:/xampp/htdocs/Piktora/.git
\item \texttt{-capture-url} http://localhost
\item \texttt{-before-capture} "php script\_piktora.php"
\end{itemize}

Terdapat masalah saat melakukan pengujian. Awalnya Opera \textit{browser} yang digunakan bukan versi \textit{portable} melainkan versi standar. Saat program dijalankan, program mengeluarkan pesan \textit{error} berupa \texttt{unknown error: cannot find Opera binary}. Setelah ditelusuri, tidak ditemukan \textit{file} "opera.exe" di dalam direktori "C:/Program Files" atau "C:/Program Files (x86)". Solusi untuk mengatasi masalah ini adalah melakukan instalasi Opera \textit{browser} versi \textit{portable} pada direktori "C:/Program Files" atau "C:/Program Files (x86)". Setelah dilakukan instalasi tersebut, program berhasil membangkitkan animasi \textit{timelapse} pada proyek Piktora menggunakan OperaDriver.

\item \textbf{Pengujian Proyek Piktora dengan EdgeDriver}\\
Pengujian pada proyek Piktora dilakukan menggunakan EdgeDriver. Pada saat melakukan pengujian, dilakukan sedikit perubahan kode program pada kelas BrowserController baris ke-46 (lihat Lampiran \ref{lamp:A}). \textit{Object} bertipe WebDriver dinisialisasi menggunakan \textit{object} bertipe EdgeDriver. Versi Microsoft Edge \textit{browser} yang digunakan untuk pengujian adalah 44.17763.1.0. Berikut ini adalah \textit{Option} yang digunakan untuk menguji program:
\begin{itemize}
\item \texttt{-project-path} C:/xampp/htdocs/Piktora/.git
\item \texttt{-capture-url} http://localhost
\item \texttt{-before-capture} "php script\_piktora.php"
\end{itemize}
Hasil pengujian ini tidak sesuai dengan harapan. Tampilan halaman \textit{web} yang ditampilkan oleh Microsoft Edge \textit{browser} saat dikontrol oleh EdgeDriver tidak rapih seperti yang diperlihatkan pada Gambar \ref{fig:layout1}. Tidak diketahui apa yang menyebabkan halaman \textit{web} tersebut menjadi tidak rapih. Sebagai perbandingan, Gambar \ref{fig:layout2} menunjukkan tampilan halaman \textit{web} yang ditampilkan oleh Microsoft Edge \textit{browser} saat tidak dikontrol oleh EdgeDriver. Meskipun tampilan dari halaman \textit{web} tidak rapih, program tetap dapat membangkitkan animasi pada proyek Piktora.    

\begin{figure}[H]
	\centering
		\includegraphics[scale=0.4]{Gambar/Layout_dengan_Edge_Driver.png}
	\caption{Tampilan halaman \textit{web} pada \textit{browser} saat tidak dikontrol oleh EdgeDriver}
	\label{fig:layout1}
\end{figure}

\begin{figure}[H]
	\centering
		\includegraphics[scale=0.4]{Gambar/Layout_tanpa_Edge_Driver.png}
	\caption{Tampilan halaman \textit{web} pada \textit{browser} saat tidak dikontrol oleh EdgeDriver.}
	\label{fig:layout2}
\end{figure}
 


\item \textbf{Pengujian Proyek Piktora dengan InternetExplorer}\\
Pengujian pada proyek Piktora dilakukan menggunakan InternetExplorerDriver. Pada saat melakukan pengujian, dilakukan sedikit perubahan kode program pada kelas BrowserController baris ke-46 (lihat Lampiran \ref{lamp:A}). \textit{Object} bertipe WebDriver dinisialisasi menggunakan \textit{object} bertipe InternetExplorerDriver. Versi Internet Explorer \textit{browser} yang digunakan untuk pengujian adalah 11.379.17763.0. Berikut ini adalah \textit{Option} yang digunakan untuk menguji program:
\begin{itemize}
\item \texttt{-project-path} C:/xampp/htdocs/Piktora/.git
\item \texttt{-capture-url} http://localhost
\item \texttt{-before-capture} "php script\_piktora.php"
\end{itemize}
Program berhasil membangkitkan animasi \textit{timelapse} pada proyek Piktora menggunakan InternetExplorerDriver.

\item \textbf{Pengujian \textit{Website} Netflix Open Source Software Center}\\
Netflix Open Source Software Center\footnote{https://netflix.github.io/} merupakan proyek Open Source yang dimiliki oleh Netflix. Repositori \textit{website} ini disimpan pada Github\footnote{https://github.com/Netflix/netflix.github.com}. Repositori ini memiliki 393 \textit{commit}. Lingkungan pengujian eksperimental ini sama dengan lingkungan implementasi yang terdapat pada subbab \ref{subsec:lingkunganimplementasi5}. Berikut ini adalah \textit{Option} yang digunakan untuk menguji program:
\begin{itemize}
\item \texttt{-project-path} :/xampp/htdocs/netflix.github.com/.git
\item \texttt{-capture-url} http://localhost
\item \texttt{-seconds-per-commit} 0.1 
\end{itemize}
Program berhasil membangkitkan animasi \textit{timelapse} dari \textit{website} Netflix Open Source Software Center. Tidak ditemukan masalah saat melakukan pengujian. Hasil dari pengujian berupa \textit{file} hasil animasi bertipe GIF dengan durasi 39 detik.


\item \textbf{Pengujian \textit{Website} Proyek Bootstrap}\\
Bootstrap\footnote{https://getbootstrap.com/} adalah kakas \textit{open source} untuk membangun \textit{website} yang dipakai bersama dengan HTML, CSS, dan JavaScript. Repositori\footnote{https://github.com/twbs/bootstrap} \textit{website} ini disimpan pada Github. Pengujian ini dilakukan pada \textit{branch} gh-pages, dimana di pada \textit{branch} tersebut terdapat 8547 \textit{commit}. Lingkungan pengujian eksperimental ini sama dengan lingkungan implementasi yang terdapat pada subbab \ref{subsec:lingkunganimplementasi5}. 
Berikut ini adalah \textit{Option} yang digunakan untuk menguji program:
\begin{itemize}
\item \texttt{-project-path} C:/xampp/htdocs/bootstrap/.git
\item \texttt{-capture-url} http://localhost
\item \texttt{-seconds-per-commit} 0.05 
\item \texttt{-branch} gh-pages
\end{itemize}

Terdapat beberapa masalah saat melakukan pengujian \textit{website} Bootstrap. Program suatu ketika berhenti dan mengeluarkan pesan error: "short SHA1 685039d is ambiguous". Pesan error ini muncul karena terdapat dua Git \textit{object} yang mempunyai 7 digit ID yang sama, sehingga tidak bisa melakukan \textit{checkout} ke \textit{commit} 685039d. Masalah berikutnya adalah perbedaan letak \textit{file} "index.html". Di beberapa \textit{commit}, \textit{file} "index.html" ini terletak di dalam direktori "docs". Karena perbedaan letak \textit{file} ini, halaman \textit{web} menjadi tidak muncul, yang muncul adalah struktur direktori dari \textit{website} Boostrap. Masalah yang terakhir yaitu di beberapa \textit{commit} tidak terdapat \textit{file} "index.html", sama seperti masalah sebelumnya hal ini menyebabkan halaman \textit{web} menjadi tidak muncul.

Solusi untuk mengatasi masalah pertama yaitu dengan mengubah kode program di kelas VCS. Awalnya program hanya menyimpan \textit{commit} ID dengan panjang 7 digit. Setelah itu kode program diubah supaya bisa menyimpan \textit{commit} ID dengan panjang 10 digit. 

Solusi untuk mengatasi masalah kedua yaitu dengan menambahkan \textit{Option} \texttt{-before-capture} saat menjalankan program. Argumen dari Option tersebut berisi \textit{terminal command} yang menjalankan \textit{script} PHP. \textit{Script} PHP ini akan mengecek letak \textit{file} "index.html" pada \textit{folder} utama dan "docs". \textit{Script} kemudian akan mengecek \textit{directory root} apache pada \textit{file} "httpd.conf". Jika \textit{directory root} sudah mengarah ke \textit{folder} tempat "index.html" berada, maka \textit{script} tidak akan mengubah isi \textit{file} "httpd.conf". Jika \textit{directory root} tidak mengarah ke \textit{folder} tempat "index.html" berada, maka \textit{script} akan mengubah \textit{directory root} pada \textit{file} "httpd.conf" dan melakukan \textit{restart} pada apache.

Untuk masalah ketiga, belum ditemukan solusinya. Jadi program akan tetap mengambil \textit{screenshot} meskipun tidak terdapat \textit{file} "index.html". Setelah menambahkan \textit{Option} \texttt{-before-capture} dan mengubah kode program sehingga menyimpan \textit{commit} ID dengan panjang 10 digit, program berhasil membangkitkan animasi. Hasil dari pengujian berupa \textit{file} hasil animasi bertipe GIF dengan durasi 7 menit   7 detik.



\end{enumerate}