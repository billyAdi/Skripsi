\documentclass[a4paper,twoside]{article}
\usepackage[T1]{fontenc}
\usepackage[bahasa]{babel}
\usepackage{graphicx}
\usepackage{graphics}
\usepackage{float}
\usepackage[cm]{fullpage}
\pagestyle{myheadings}
\usepackage{etoolbox}
\usepackage{setspace} 
\usepackage{lipsum} 
\setlength{\headsep}{30pt}
\usepackage[inner=2cm,outer=2.5cm,top=2.5cm,bottom=2cm]{geometry} %margin
% \pagestyle{empty}

\makeatletter
\renewcommand{\@maketitle} {\begin{center} {\LARGE \textbf{ \textsc{\@title}} \par} \bigskip {\large \textbf{\textsc{\@author}} }\end{center} }
\renewcommand{\thispagestyle}[1]{}
\markright{\textbf{\textsc{AIF401/AIF402 \textemdash Rencana Kerja Skripsi \textemdash Sem. Ganjil 2018/2019}}}

\onehalfspacing
 
\begin{document}

\title{\@judultopik}
\author{\nama \textendash \@npm} 

%tulis nama dan NPM anda di sini:
\newcommand{\nama}{Billy Adiwijaya}
\newcommand{\@npm}{2015730053}
\newcommand{\@judultopik}{Pembangkit Timelapse Pengembangan Proyek Perangkat Lunak} % Judul/topik anda
\newcommand{\jumpemb}{1} % Jumlah pembimbing, 1 atau 2
\newcommand{\tanggal}{19/08/2018}
\maketitle

\pagenumbering{arabic}

\section{Deskripsi}
	Git adalah salah satu perangkat lunak \textit{version control system} yang digunakan untuk pengembangan proyek perangkat lunak. Perangkat lunak yang tersimpan pada \textit{git repository} dapat dipantau perkembangannya, mulai dari awal pengembangan proyek hingga proyek selesai. Perubahan yang terjadi pada \textit{repository} dicatat oleh Git dalam bentuk \textit{commit}. Setiap \textit{commit} mengandung informasi mengenai perubahan yang terjadi pada \textit{repository}, waktu perubahan, dan orang yang melakukan perubahan. \textit{Database} pada \textit{git} tidak bersifat terpusat. Setiap orang yang terlibat mempunyai \textit{database} git masing-masing, sehingga pengelolaan perangkat lunak dapat dilakukan secara daring dan luring. 

Selenium WebDriver adalah perangkat lunak yang digunakan untuk melakukan \textit{automated testing} pada perangkat lunak berbasis web. Dengan menggunakan Selenium WebDriver, pengguna dapat memasukkan \textit{script} bahasa pemrograman tertentu untuk melakukan pengujian. Bahasa pemrograman yang didukung yaitu C\#, Java, Perl, PHP, Python, Ruby, dan JavaScript. Selenium WebDriver dapat melakukan pengujian pada \textit{browser} Google Chrome, Mozilla Firefox, Opera, Safari, dan Internet Explorer.  
  
Pada skripsi ini, akan dibuat sebuah perangkat lunak yang dapat menampilkan animasi timelapse dari pengembangan proyek perangkat lunak. Perangkat lunak ini dibangun menggunakan bahasa Java. Perangkat lunak ini menggunakan tampilan terminal/konsol. Dalam pembuatan animasi timelapse, dibutuhkan perangkat lunak Selenium WebDriver dan JGit.
 

\section{Rumusan Masalah}
Rumusan masalah dari skripsi ini adalah sebagai berikut:
\begin{enumerate}
	\item Bagaimana cara membangkitkan animasi timelapse pada pengembangan proyek perangkat lunak?
	\item Bagaimana cara menampilkan animasi timelapse?
\end{enumerate}

\section{Tujuan}
Tujuan dari skripsi ini adalah sebagai berikut:
\begin{enumerate}
	\item Mengetahui cara untuk membangkitkan animasi timelapse pada pengembangan proyek perangkat lunak.
	\item Mengetahui cara untuk menampilkan animasi timelapse. 
\end{enumerate}

\section{Deskripsi Perangkat Lunak}
Perangkat lunak akhir yang akan dibuat memiliki fitur minimal sebagai berikut:
\begin{itemize}
	\item Pengguna dapat memasukkan alamat web dari proyek perangkat lunak.   
	\item Perangkat lunak dapat membangkitkan animasi timelapse pada pengembangan proyek perangkat lunak.
	\item Perangkat lunak dapat menampilkan animasi timelapse. 
\end{itemize}

\section{Detail Pengerjaan Skripsi}
Bagian-bagian pekerjaan skripsi ini adalah sebagai berikut :
	\begin{enumerate}
		\item Melakukan studi literatur tentang Selenium WebDriver dan JGit.
		\item Melakukan analisis penggunaan Selenium WebDriver dan JGit untuk membangkitkan animasi.
		\item Merancang perangkat lunak.
		\item Membangun perangkat lunak.
		\item Melakukan eksperimen dan pengujian pada perangkat lunak.
		\item Menulis dokumen skripsi.
	\end{enumerate}

\section{Rencana Kerja}
\begin{center}
  \begin{tabular}{ | c | c | c | c | l |}
    \hline
    1*  & 2*(\%) & 3*(\%) & 4*(\%) &5*\\ \hline \hline
    1   & 15  & 15  &  &  \\ \hline
    2   & 15 & 15  &   & \\ \hline
    3   & 15  &   & 15 & \\ \hline
    4   & 15  &   & 15 & \\ \hline
    5   & 20  &   & 20 & \\ \hline
    6   & 20  & 10  & 10 &  {\footnotesize menulis dokumen skripsi hingga bab 3 pada S1}\\ \hline
    Total  & 100  & 40  & 60 &  \\ \hline
                          \end{tabular}
\end{center}

Keterangan (*)\\
1 : Bagian pengerjaan Skripsi (nomor disesuaikan dengan detail pengerjaan di bagian 5)\\
2 : Persentase total \\
3 : Persentase yang akan diselesaikan di Skripsi 1 \\
4 : Persentase yang akan diselesaikan di Skripsi 2 \\
5 : Penjelasan singkat apa yang dilakukan di S1 (Skripsi 1) atau S2 (skripsi 2)

\vspace{1cm}
\centering Bandung, \tanggal\\
\vspace{2cm} \nama \\ 
\vspace{1cm}

Menyetujui, \\
\ifdefstring{\jumpemb}{2}{
\vspace{1.5cm}
\begin{centering} Menyetujui,\\ \end{centering} \vspace{0.75cm}
\begin{minipage}[b]{0.45\linewidth}
% \centering Bandung, \makebox[0.5cm]{\hrulefill}/\makebox[0.5cm]{\hrulefill}/2013 \\
\vspace{2cm} Nama: \makebox[3cm]{\hrulefill}\\ Pembimbing Utama
\end{minipage} \hspace{0.5cm}
\begin{minipage}[b]{0.45\linewidth}
% \centering Bandung, \makebox[0.5cm]{\hrulefill}/\makebox[0.5cm]{\hrulefill}/2013\\
\vspace{2cm} Nama: \makebox[3cm]{\hrulefill}\\ Pembimbing Pendamping
\end{minipage}
\vspace{0.5cm}
}{
% \centering Bandung, \makebox[0.5cm]{\hrulefill}/\makebox[0.5cm]{\hrulefill}/2013\\
\vspace{2cm} Nama: \makebox[3cm]{\hrulefill}\\ Pembimbing Tunggal
}

\end{document}

