\documentclass[a4paper,twoside]{article}
\usepackage[T1]{fontenc}
\usepackage[bahasa]{babel}
\usepackage{graphicx}
\usepackage{graphics}
\usepackage{float}
\usepackage[cm]{fullpage}
\pagestyle{myheadings}
\usepackage{etoolbox}
\usepackage{setspace} 
\usepackage{lipsum} 
\setlength{\headsep}{30pt}
\usepackage[inner=2cm,outer=2.5cm,top=2.5cm,bottom=2cm]{geometry} %margin
% \pagestyle{empty}

\makeatletter
\renewcommand{\@maketitle} {\begin{center} {\LARGE \textbf{ \textsc{\@title}} \par} \bigskip {\large \textbf{\textsc{\@author}} }\end{center} }
\renewcommand{\thispagestyle}[1]{}
\markright{\textbf{\textsc{AIF401/AIF402 \textemdash Rencana Kerja Skripsi \textemdash Sem. Ganjil 2018/2019}}}

\newcommand{\HRule}{\rule{\linewidth}{0.4mm}}
\renewcommand{\baselinestretch}{1}
\setlength{\parindent}{0 pt}
\setlength{\parskip}{6 pt}

\onehalfspacing
 
\begin{document}

\title{\@judultopik}
\author{\nama \textendash \@npm} 

%tulis nama dan NPM anda di sini:
\newcommand{\nama}{Billy Adiwijaya}
\newcommand{\@npm}{2015730053}
\newcommand{\@judultopik}{Pembangkit Timelapse Pengembangan Proyek Perangkat Lunak Berbasis Web} % Judul/topik anda
\newcommand{\jumpemb}{1} % Jumlah pembimbing, 1 atau 2
\newcommand{\tanggal}{03/09/2018}

% Dokumen hasil template ini harus dicetak bolak-balik !!!!

\maketitle

\pagenumbering{arabic}

\section{Deskripsi}
 Git adalah salah satu perangkat lunak \textit{version control system} yang digunakan untuk pengembangan proyek perangkat lunak. Perangkat lunak yang tersimpan pada \textit{git repository} dapat dipantau perkembangannya, mulai dari awal pengembangan proyek hingga proyek selesai. Perubahan yang terjadi pada \textit{repository} dicatat oleh Git dalam bentuk \textit{commit}. Setiap \textit{commit} mengandung informasi mengenai perubahan yang terjadi pada \textit{repository}, waktu perubahan, dan orang yang melakukan perubahan. \textit{Database} pada \textit{git} tidak bersifat terpusat. Setiap orang yang terlibat mempunyai \textit{database} git masing-masing, sehingga pengelolaan perangkat lunak dapat dilakukan secara \textit{online} dan \textit{offline}. 

Selenium adalah seperangkat alat yang secara khusus digunakan untuk mengotomatisasi \textit{web browsers}. Dengan menggunakan Selenium WebDriver, pengguna dapat memasukkan \textit{script} bahasa pemrograman tertentu untuk melakukan pengujian. Bahasa pemrograman yang didukung yaitu C\#, Java, Perl, PHP, Python, Ruby, dan JavaScript. Selenium WebDriver dapat melakukan pengujian pada \textit{browser} Google Chrome, Mozilla Firefox, Opera, Safari, dan Internet Explorer.  
  
Pada skripsi ini, akan dibuat sebuah perangkat lunak yang dapat menampilkan animasi \textit{timelapse} dari pengembangan proyek perangkat lunak berbasis web. Perangkat lunak ini dibangun menggunakan bahasa Java. Perangkat lunak ini menggunakan tampilan terminal/konsol. Dalam pembuatan animasi \textit{timelapse}, dibutuhkan \textit{library} Selenium WebDriver dan JGit. 
\section{Rumusan Masalah}
Rumusan masalah dari skripsi ini adalah sebagai berikut:
\begin{enumerate}
	\item Bagaimana cara membangkitkan animasi \textit{timelapse} pada pengembangan proyek perangkat lunak berbasis web?
	\item Bagaimana cara mengimplementasikan aplikasi untuk membangkitkan \textit{timelapse} pada pengembangan proyek perangkat lunak berbasis web?
\end{enumerate}
\section{Tujuan}
Tujuan dari skripsi ini adalah sebagai berikut:
\begin{enumerate}
	\item Mengetahui cara untuk membangkitkan animasi \textit{timelapse} pada pengembangan proyek perangkat lunak berbasis web.
	\item Mengetahui cara untuk mengimplementasikan aplikasi untuk membangkitkan \textit{timelapse} pada pengembangan proyek perangkat lunak berbasis web. 
\end{enumerate} 
\section{Deskripsi Perangkat Lunak}
Perangkat lunak akhir yang akan dibuat memiliki fitur minimal sebagai berikut:
\begin{itemize}
	\item Pengguna dapat memasukkan alamat direktori dari proyek perangkat lunak berbasis web yang terekam oleh Git.
	\item Perangkat lunak dapat membaca konfigurasi dari parameter.
	\item Perangkat lunak dapat menelusuri \textit{history} perkembangan perangkat lunak berbasis web dengan fitur Git.   
	\item Perangkat lunak dapat membangkitkan animasi \textit{timelapse} pada pengembangan proyek perangkat lunak berbasis web.
	 
\end{itemize}
\section{Detail Pengerjaan Skripsi}
Bagian-bagian pekerjaan skripsi ini adalah sebagai berikut :
	\begin{enumerate}
		\item Melakukan studi literatur tentang Selenium WebDriver, Git, dan JGit.
		\item Melakukan analisis penggunaan Selenium WebDriver dan JGit untuk membangkitkan animasi \textit{timelapse}.
		\item Merancang perangkat lunak.
		\item Membangun perangkat lunak.
		\item Melakukan eksperimen dan pengujian pada perangkat lunak.
		\item Menulis dokumen skripsi.
	\end{enumerate}

\section{Rencana Kerja}
Rincian capaian yang direncanakan di Skripsi 1 adalah sebagai berikut:
\begin{enumerate}
\item Melakukan studi literatur tentang Selenium WebDriver, Git, dan JGit.
\item Melakukan analisis penggunaan Selenium WebDriver dan JGit untuk membangkitkan animasi \textit{timelapse}.
\item Menulis dokumen skripsi.
\end{enumerate}

Sedangkan yang akan diselesaikan di Skripsi 2 adalah sebagai berikut:
\begin{enumerate}
\item Merancang perangkat lunak.
\item Membangun perangkat lunak.
\item Melakukan eksperimen dan pengujian pada perangkat lunak.
\item Menulis dokumen skripsi. 
\end{enumerate}

\vspace{1cm}
\centering Bandung, \tanggal\\
\vspace{2cm} \nama \\ 
\vspace{1cm}

Menyetujui, \\
\ifdefstring{\jumpemb}{2}{
\vspace{1.5cm}
\begin{centering} Menyetujui,\\ \end{centering} \vspace{0.75cm}
\begin{minipage}[b]{0.45\linewidth}
% \centering Bandung, \makebox[0.5cm]{\hrulefill}/\makebox[0.5cm]{\hrulefill}/2013 \\
\vspace{2cm} Nama: \makebox[3cm]{\hrulefill}\\ Pembimbing Utama
\end{minipage} \hspace{0.5cm}
\begin{minipage}[b]{0.45\linewidth}
% \centering Bandung, \makebox[0.5cm]{\hrulefill}/\makebox[0.5cm]{\hrulefill}/2013\\
\vspace{2cm} Nama: \makebox[3cm]{\hrulefill}\\ Pembimbing Pendamping
\end{minipage}
\vspace{0.5cm}
}{
% \centering Bandung, \makebox[0.5cm]{\hrulefill}/\makebox[0.5cm]{\hrulefill}/2013\\
\vspace{2cm} Nama: \makebox[3cm]{\hrulefill}\\ Pembimbing Tunggal
}
\end{document}

