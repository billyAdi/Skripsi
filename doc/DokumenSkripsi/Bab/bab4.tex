\chapter{Perancangan}
\label{chap:Perancangan}

Pada bab ini dijelaskan mengenai perancangan perangkat lunak yang dibangun, meliputi perancangan kelas dan perancangan antarmuka.

\section{Perancangan Kelas}
\label{sec:perancangan_kelas}
Berikut adalah kelas yang terdapat pada perangkat lunak pembangkit \textit{timelapse}:
\begin{itemize}
\item SeleniumWebDriver\\
Kelas ini berfungsi untuk membuka halawan \textit{web}, mengambil \textit{screenshot}, dan menyimpan \textit{file} hasil \textit{screenshot}. Berikut adalah atribut yang terdapat pada kelas ini:
\begin{itemize}
   \item private WebDriver driver\\
   Atribut ini berfungsi untuk berinteraksi dengan WebDriver. WebDriver merupakan sebuah \textit{browser} yang digunakan untuk keperluan \textit{automation testing}.
    \item private ArrayList<File> fileScreenshot\\
	Atribut ini berfungsi untuk menyimpan \textit{file} hasil \textit{screenshot}.
\end{itemize}

Berikut adalah \textit{method} yang terdapat pada kelas ini:
\begin{itemize}
   
  \item public void changePage(String url)//
  Berfungsi untuk berpindah halaman pada \textit{browser}. Parameternya adalah alamat url yang akan dibuka.
  \item public void quit()\\
  Berfungsi untuk menutup \textit{browser}.
  \item public void takeScreenshot()\\
  Berfungsi untuk mengambil \textit{screenshot} pada \textit{browser} dan menyimpannya ke atribut fileScreenshot.
\end{itemize}

\item CommandLineOptions\\
Kelas ini berfungsi untuk menyimpan semua \textit{option} yang terdapat dalam program ini, dan menguraikan argumen Command Line Options yang dimasukkan oleh \textit{user}.  
Berikut adalah atribut yang terdapat pada kelas ini:
\begin{itemize}
	\item private Options options\\
	Atribut ini berfungsi untuk menyimpan semua \textit{option} yang terdapat dalam program ini.
    \item private CommandLineParser parser\\
    Atribut ini berfungsi untuk menguraikan argumen dari Command Line Options.
    \item private CommandLine commandLine\\
    Atribut ini berfungsi untuk menampung argumen hasil \textit{parsing} dari atribut parser.
    \item private Runtime rt\\
    Bersama dengan atribut proc berfungsi untuk menjalankan \textit{script} PHP.
    \item private Process proc\\
    Bersama dengan atribut rt berfungsi untuk menjalankan \textit{script} PHP.
\end{itemize}
Berikut adalah \textit{method} yang terdapat pada kelas ini:
\begin{itemize}
\item public String getOptionValue(String optionName)\\
Berfungsi untuk mendapatkan nilai argumen pada suatu \textit{option}. Parameternya adalah nama \textit{option}.
\item public void runScript()\\
Berfungsi untuk menjalankan \textit{script} PHP.
\end{itemize}
\item VCS\\
Kelas ini digunakan untuk berinteraksi pada proyek perangkat lunak yang terekam oleh Git. Interaksi dilakukan menggunakan operasi yang terdapat pada Git. Berikut adalah atribut yang terdapat pada kelas ini:
\begin{itemize}
   \item private Git git\\
   Atribut ini digunakan untuk melakukan interaksi pada proyek perangkat lunak yang terekam oleh Git.
	\item private RevWalk revWalk\\
	Atribut ini digunakan untuk menelusuri histori \textit{commit}.
    \item private ArrayList<String> commitID\\
    Atribut ini digunakan untuk menampung seluruh \textit{commit} ID dari hasil penelusuran histori.
    \item private Repository repo\\
    Atribut ini merepresentasikan sebuah repositori Git. Atribut ini juga digunakan sebagai parameter dari \textit{constructor} atribut git.
\end{itemize}

Berikut ini adalah \textit{method} yang terdapat dalam kelas ini:
\begin{itemize}
\item public void getHistoryCommit()\\
Berfungsi untuk mendapatkan seluruh histori \textit{commmit}.
\item public void checkoutCommit(int indexCommit)\\
Berfungsi untuk melakukan \textit{checkout} ke \textit{commit} tertentu. Parameter dari \textit{method} ini adalah index \textit{commit} tujuan.
\item public void checkoutMaster()\\
Berfungsi untuk melakukan \textit{checkout} ke \textit{commit} terakir.
\item public void reset()\\
Berfungsi untuk menghapus perubahan pada \textit{working tree} di \textit{commit} tertentu. \textit{Method} ini dipanggil setelah \textit{script} PHP dijalankan. 
\item public int getCommitSize()\\
Berfungi untuk mendapatkan panjang dari atribut commitID.
\item public int getIndexCommit(String idCommit)\\
Berfungsi untuk mendapatkan index \textit{commit} dari atribut commitID berdasarkan ID \textit{commit}. 
\end{itemize}


\item TimeLapseGenerator\\
Kelas ini digunakan untuk membangkitkan animasi \textit{timelapse}. Hasil dari animasi berupa \textit{file} GIF.
Berikut adalaah atribut yang dimiliki oleh kelas ini:
\begin{itemize}
\item private CommandLineOptions commandLineoptions\\
Fungsinya sama seperti kelas CommandLineOptions.
\item private VCS vcs\\
Fungsinya sama seperti kelas VCS.
\item private SeleniumWebDriver seleniumWebDriver\\
Fungsinya sama seperti kelas SeleniumWebDriver.
\item private ArrayList<File> fileScreenshot\\
Atribut berfungsi untuk menampung hasil \textit{screenshot}.
\end{itemize}

Berikut adalah \textit{method} yang dimiliki oleh kelas ini:
\begin{itemize}
\item public void generateTimelapse()\\
Berfungsi untuk membangkitkan animasi \textit{timelapse} berdasarkan langkah-langkah pada Bab \ref{subsec:langkah_animasi}. Hasil animasi berupa \textit{file} bertipe GIF.
\end{itemize}

\end{itemize}

\section{Perancangan Antarmuka}
\label{sec:perancangan_antarmuka}